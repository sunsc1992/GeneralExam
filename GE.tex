\documentclass[12pt]{article}
\usepackage{achemso}
\usepackage[margin=1.0in]{geometry}
\usepackage{setspace}
\usepackage{amsmath}             % for equation typesetting
\usepackage{amssymb}             % for equation typesetting
\usepackage{mathrsfs} 
\usepackage{wasysym}             % for geometric shapes
\usepackage{color}               % for colored fonts
\usepackage{setspace}            % for 1.5 and double spacing
\usepackage{graphicx}            % main graphics package
\usepackage{wrapfig}             % allow text wrapping around figures
\doublespacing

\usepackage{appendix}

\usepackage[capitalize]{cleveref}
\crefname{figure}{Fig.}{Figs.}
\Crefname{figure}{Figure}{Figures}
\crefname{table}{Tab.}{Tabs.}
\Crefname{table}{Table}{Tables}
\crefname{equation}{Eq.}{Eqs.}
\Crefname{equation}{Equation}{Equations}
\crefname{section}{Sec.}{Secs.}
\Crefname{section}{Section}{Sections}
\crefname{appsec}{appendix}{appendices}
\Crefname{appsec}{Appendix}{Appendixes}

% Mathematical Shortcuts
\newcommand{\pfrac}[2]{\frac{\partial #1}{\partial #2}}   % partial derivative
\newcommand{\difrac}[2]{\frac{d #1}{d #2}}                % derivative
\newcommand{\bpar}[1]{\left( #1 \right)}                  % big parentheses
\newcommand{\bbra}[1]{\left[ #1 \right]}                  % big brackets
\newcommand{\bbar}[1]{\left| #1 \right|}                  % big bars
\newcommand{\bra}[1]{\left\langle #1 \right\vert}         % bra
\newcommand{\ket}[1]{\left\vert #1 \right\rangle}         % ket
\newcommand{\inner}[2]{\left\langle #1 \left\vert\right. #2 \right\rangle}            % bracket
\newcommand{\innerop}[3]{\left\langle #1 \left\vert #2 \right\vert #3 \right\rangle}  % operator matrix element
\newcommand{\innersub}[4]{\langle \bd{#1}_{#2}, \bd{#3}_{#4} \rangle}                 % bracket with subscripts
\newcommand{\half}{\frac{1}{2}}                           % 1/2
\newcommand{\powfrac}[3]{\bpar{\frac{#1}{#2}}^{#3}}       % fraction raised to a power
\newcommand{\ii}{\infty}                                  % infinity symbol
\newcommand{\tquad}{\quad\quad\quad}                      % triple-quad spacing
\renewcommand{\Im}{\text{Im}}                             % imaginary symbol
\renewcommand{\Re}{\text{Re}}                             % real symbol

\newcommand*\mycommand[1]{\texttt{\emph{#1}}}
\newcommand*\suchthat[0]{\text{ }\vert\text{ }}
\newcommand*\complexmatfield[1]{\mathbb{C}^{#1\text{x}#1}}
\newcommand*\speciallinear[2]{\mathrm{UnL}(#1,#2)}
\newcommand*\complexspeciallinear[1]{\speciallinear{#1}{\mathbb{C}}}
\newcommand*\generallinear[2]{\mathrm{GL}(#1,#2)}
\newcommand*\complexgenerallinear[1]{\generallinear{#1}{\mathbb{C}}}
\newcommand*\mat[1]{\boldsymbol{#1}}

\newcommand*\vc[1]{\boldsymbol{#1}}


\title{General Exam}
\date{October 19, 2016 \\ CHB 239}
\author{David Williams-Young\\ Department of Chemistry, University of Washington}


\begin{document}
\maketitle


\newpage
\section{Introduction}

Electronic and nuclear motion are fundamental to our understanding of chemical
and physical phenomena. From electronic reorganization upon photo-induced charge
transfer in the chromophores of solar cells to the ultrafast excitonic dynamics
in nanocrystalline materials, molecular dynamics lies at the heart of chemistry.
As such, to properly model these phenomena theoretically, we must often venture
into the time-domain to capture the physics necessary to completely understand
the problem at hand. As such, this task constitutes a primary focus of modern
\emph{ab initio} quantum molecular theory. \textbf{In this work, I will outline
my proposed research to accurately simulate \emph{ab initio} relativistic
molecular dynamics using correlated multi--configurational wave functions}.

In the quantum mechanical description of molecular systems, neglecting explicit
coupling to a quantized photon field, time evolution of the total molecular wave
function, $\ket{\Psi (t)}$ is governed by the Hamiltonian wave equation 
(in atomic units),
\begin{equation}
\mathscr{H}(t) \ket{\Psi (t)} = i\partial_t \ket{\Psi(t)} \quad,
\label{eq:WaveEq}
\end{equation}
where $\mathscr{H}(t)$ is the time--dependent Hamiltonian and $\partial_t$ is a
partial derivative with respect to time, and $t$ is a time variable. We have
adopted the standard description of the wave function as a state vector rather
than an explicit function of coordinates. The moieties enclosed in parentheses
are taken to be parameters, i.e. the wave function is parameterized by time.
In principle, one may solve \cref{eq:WaveEq} simultaneously for both the 
electronic and nuclear degrees of freedom explicitly. This approach is, however,
intractable for quantum systems exceeding more than a few particles. To simplify
the solutions of \cref{eq:WaveEq}, one may formally decompose $\ket{\Psi (t)}$
into a product of nuclear and electronic wave functions,
\begin{equation} 
\ket{\Psi (t)} = \ket{\Phi(\vc{R}(t),t)}\otimes\ket{\Theta(t)} 
\quad .  
\label{eq:exactSepElecNuc}
\end{equation} 
%Here, $\vc{r}$ and $\vc{R}$ are the electronic and nuclear coordinates
%respectively, $t$ is a time parameter, and $\Phi(\vc{r} \vert \vc{R}, t)$ and
%$\Theta(\vc{r} \vert t)$ are the time--dependent electronic and nuclear wave
%functions respectively. The vertical separator found in the parameters of the
%above wave functions are used to denote a parametric time dependence of the
%function on those parameters that appear to the right of the separator. 
Here, $\ket{\Phi(\vc{R}(t),t)}$ and $\ket{\Theta (t)}$ are the electronic and
nuclear wave functions respectively and $\vc{R}(t)$ is the expectation value of
the nuclear position operator, $\vc{R}(t) =
\innerop{\Theta(t)}{\hat{\vc{R}}}{\Theta(t)}$.  It should be noted that the
separation of the total molecular wave function in \cref{eq:exactSepElecNuc} is
formally exact\cite{Gross10_PRL123002, Cederbaum08_JCP124101} without loss of
generality in the non-relativistic treatment of quantum mechanics. Even so, the
equations of motion for the time dependence of \cref{eq:exactSepElecNuc} are
wildly complicated\cite{Ghosh15_MP1} and would only be practical for molecules
consisting of a few atoms.  Nevertheless, this explicitly exact treatment allows
one to, in principle, bifurcate the treatment of quantum molecular dynamics into
explicit treatment of the electronic and nuclear time dependence separately.
Thus, in order to theoretically treat molecular dynamics accurately, one must be
able to treat both the electronic and nuclear degrees of freedom to some level
of accuracy.  This separation will be the underlying principle of the future
directions that I propose in this work.

Over the years, perhaps the most successful \emph{ab initio} treatment of the
electronic degrees of freedom of the total wave function has been in the
configuration interaction (CI) expansion of the electronic wave function. In the
CI expansion, the exact electronic wave function is written as a linear
combination of Slater determinants,
\begin{equation}
\ket{\Phi (\vc{R}(t),t)} = \sum_i  \ket{\psi_i (\vc{R}(t),t)} c_i(t)
\quad ,
\label{eq:CIExp}
\end{equation}
where $\left\lbrace\ket{\psi_i}\right\rbrace$ is a set of Slater determinantal
electronic wave functions and $\left\lbrace c_i \right\rbrace$ is a set of
expansion coefficients.  In principle, the above expansion is exact given that
all possible electronic configurations are included. Given that each of the
Slater determinants is written as a linear combination of single particle Fock
states, $\left\lbrace \phi_p \right\rbrace$, hereafter referred to as
\emph{molecular orbitals}, there are a total of $\mathcal{C}(N,K)$ possible
determinants that would need to be included in the expansion, where $N$ and $K$
are the total number of electrons and the number of molecular orbitals in
$\left\lbrace \phi_p \right\rbrace$, respectively. Such an exhaustive expansion
is called \emph{full configuration interaction} (FCI) and is rendered
impractical for the majority of systems due to the rapid growth of the total
number of required Slater determinants.

In practice, we  must restrict which Slater determinants we include in
\cref{eq:CIExp}. As a result, a vast proportion of research in the field of
electronic structure theory has been centered on the proper way to restrict the
expansion in such a way that all of the relevant physics is maintained while
extraneous information may be discarded. Such methods as complete active space
(CAS), coupled cluster (CC), multireference configuration interaction (MRCI),
density matrix renormalization group (DMRG), particle--particle propagator
methods (pp-RPA/TDA), and truncated CI methods may all be described, at their
core, as algorithms to provide such a filtering of the physically relevant
determinants of the CI expansion. While each of these methods have their faults
and accolades, whether it be accuracy or computational burden, the future
directions proposed in this work will primarily focus on CAS and MRCI methods,
which will be explained further in \cref{sec:Future}, though the pp-RPA/TDA
methods will be discussed to some extent in \cref{sec:pp-X2C,sec:pp-TSH}.

To properly treat the nuclear degrees of freedom, there exist a vast plethora
of computational methods to treat the nuclear dynamics to varying degrees of
accuracy. In the proposed work, I will restrict the discussion of these methods
to only include those that are semiclassical. Semiclassical nuclear dynamics
algorithms are those that treat the nuclear motion classically (by some variant
of Newton's equations of motion) while maintaining a quantum description of the
electronic degrees of freedom. That is to say that we may write the nuclear 
wave function as a product of delta functions centered at the classical nuclear
positions,
\begin{equation}
\inner{\vc{R}}{\Theta (t)} = 
  \prod_{A = 0}^{N_\mathrm{atoms}} \delta^3(\vc{R} - \vc{R}_A(t))
  \quad,
  \label{eq:ClassicalNuclei}
\end{equation}
where $N_\mathrm{atoms}$ is the number of nuclei in the molecule.  This is
often a valid approach, as the nuclei of molecules are orders of magnitude
heavier than the electrons, and thus move much slower. That is not to say that
nuclei are not, strictly speaking, quantized in their time evolution, but
rather that a large proportion of quantum effects in full molecular dynamics
are due to the time dependence of the electrons. Thus it is usually a valid
assumption that the nuclei time evolve classically relative to their electronic
counterparts. That being said, there are cases in which the quantum evolution
of the nuclei must be explicitly treated, such as tunneling in lighter atoms,
in which these semiclassical methods will not suffice to capture the necessary
physics.

Within the semiclassical treatment of molecular dynamics, computational methods
may be further classified as either adiabatic or non-adiabatic. Both of these
classifications may be characterized by their treatment of the transversal of
the manifold of energy eigenstates,
\begin{equation}
\mathscr{H}(t) \ket{\Psi_I (t)} = E_I(t) \ket{\Psi_I (t)}
\quad \forall t.
\label{eq:EnergyEig}
\end{equation}
Where the index $I$ labels a particular energy eigenstate while $E_I$ represents
the energy eigenvalue of that state. We will assume that $E_I(t)$ is a smooth
function of $t$ for the sake simplicity. Adiabatic time evolution is rooted in
the adiabatic theorem in that, given an initial state that exists as an energy
eigen state of $\mathscr{H}(0)$, $\ket{\Psi_I(0)}$, at some time $t' > 0$ in the
future, the quantum system will remain in an eigenstate of the
\emph{corresponding} eigenstate of $\mathscr{H}(t')$, $\ket{\Psi_I(t')}$. In
order for this approximation to be valid (although it can be proven that this
assumption is a best an approximation while an exact adiabatic time evolution
cannot exist unless the quantum system does not time-evolve), it is a
necessary and sufficient condition that the individual energy eigen states do
not couple though the time-derivative operator,
\begin{equation}
\innerop{\Psi_I(t)}{\partial_t}{\Psi_J (t)}
  \stackrel{\mathrm{\mbox{\footnotesize adiabatic}}}{=} 0
  \quad \forall I,J,t.
  \label{eq:AdiabaticNAC}
\end{equation}
\Cref{eq:AdiabaticNAC} may be interpreted as an inability to transport
information between energy eigenstates within the adiabatic approximation and
thus the inability to leave said eigenstate throughout the time--evolution.  In
the field of \emph{ab initio} semiclassical dynamics, the workhorses of
adiabatic time evolution are Born--Oppenheimer  (BOMD) and Car--Parinello (CPMD)
molecular dynamics. In the proposed work, I will primarily focus on the BOMD
method as a choice for adiabatic time--evolution which will be discussed in
\cref{sec:Future}.  As it should logically follow, non-adiabatic dynamics may be
characterized by the full inclusion of terms like the left-hand side of
\cref{eq:AdiabaticNAC}, i.e.  it is in general not possible or practical to
ignore the coupling between energy eigenstates. As such it is possible to
transverse the manifold of energy eigenstates throughout the time evolution.
In this proposed work, I will limit the discussion of non--adiabatic dynamics to
that of the trajectory surface hopping (TSH) method which will be discussed at
some length in \cref{sec:pp-TSH,sec:Future}.

Recent years have also seen new developments in the realm of relativistic
quantum chemistry.  Relativistic effects, while often neglected in most standard
treatment of electronic structure, can have profound consequences in chemical
systems.\cite{Pyykko12_45} Scalar relativistic effects cause the contraction of
the core electron shells of heavy atoms, but perhaps of even more consequence is
the introduction of spin couplings in the Hamiltonian.  Spin-spin and spin-orbit
interactions can affect the electronic spin dynamics even in light atoms, and a
direct consequence of these couplings on excited states is the loss of
degeneracies of spin-eigenstates, giving rise to fine structure splittings (FSS)
in atoms and molecules with symmetry-induced degeneracies. It is therefore
desirable to develop accurate and cost-effective relativistic electronic
structure methods able to model such effects.

%This work is organized into four sections.
This work is organized into three subsequent sections.
%\cref{sec:Chebyshev} details the work I've contributed to accelerating
%electronic dynamics through the Chebyshev expansion of the quantum propagator
%in real-time density functional theory.
\cref{sec:pp-X2C} focuses on my recent work of the extension of the
particle-particle Tamm-Dancoff approximation to relativistic Hamiltonians.
\cref{sec:pp-TSH} describes my contribution to semi-classical non-adiabatic
molecular dynamics through the application of the particle-particle Tamm-Dancoff
approximation to the trajectory surface hopping method.  Finally,
\cref{sec:Future} outlines my plan for future development in the field of
electronic structure and dynamics in the relativistic treatment of adiabatic and
non-adiabatic phenomena in molecular systems.

%\section{The Chebyshev Expansion of the Quantum Propagator}
%\label{sec:Chebyshev}

% Relativistic pp-TDA
\section{The Relativistic Particle-Particle Tamm-Dancoff Approximation}
\label{sec:pp-X2C}


% TSH with pp-TDA
\section{Trajectory Surface Hopping within the Particle-Particle Tamm-Dancoff Approximation}
\label{sec:pp-TSH}
Test\cite{DBWY16_JCTC935,DBWY16_Accepted1,DBWY16_Submitted1,DBWY16_Submitted2}

% Future
\section{Future Directions: Relativistic Electronic Dynamics with Multi-Configurational Wave Functions}
\label{sec:Future}

\bibliography{Journal_Short_Name,ppSH,Li_Group_References,GE,DBWY}
\end{document}
