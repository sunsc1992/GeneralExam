\documentclass[12pt]{article}
\usepackage{achemso}
\usepackage[margin=1.0in]{geometry}
\usepackage[capitalize]{cleveref}
\usepackage{setspace}
\doublespacing

\usepackage{appendix}
\crefname{figure}{Fig.}{Figs.}
\Crefname{figure}{Figure}{Figures}
\crefname{table}{Tab.}{Tabs.}
\Crefname{table}{Table}{Tables}
\crefname{equation}{Eq.}{Eqs.}
\Crefname{equation}{Equation}{Equations}
\crefname{section}{Sec.}{Secs.}
\Crefname{section}{Section}{Sections}
\crefname{appsec}{appendix}{appendices}
\Crefname{appsec}{Appendix}{Appendixes}


\title{General Exam}
\date{October 19, 2016 \\ CHB 239}
\author{David Williams-Young\\ Department of Chemistry, University of Washington}


\begin{document}
\maketitle


\newpage
\section{Introduction}

This work is organized into four sections.
\cref{sec:Chebyshev} details the work I've contributed to accelerating 
electronic dynamics through the Chebyshev expansion of the quantum propagator 
in real-time density functional theory.
\cref{sec:pp-X2C} focuses on my recent work of the extension of the
particle-particle Tamm-Dancoff approximation to relativistic Hamiltonians.
\cref{sec:pp-TSH} describes my contribtion to semi-classical non-adiabatic 
molecular dynamics through the application of the particle-particle Tamm-Dancoff
approximation to the trajectory surface hopping method.
Finally, \cref{sec:Future} outlines my plan for future development in the field
of electronic structure and dynamics in the relativistic treament of non-adiabatic
phenomena in molecular systems.

\section{The Chebyshev Expansion of the Quantum Propagator}
\label{sec:Chebyshev}
\section{The Relativistic Particle-Particle Tamm-Dancoff Approximation}
\label{sec:pp-X2C}


% TSH with pp-TDA
\section{Trajectory Surface Hopping within the Particle-Particle Tamm-Dancoff Approximation}
\label{sec:pp-TSH}

Trajectory surface-hopping (TSH) is one of the most widely applied methods for
simulating electronically non-adiabatic dynamics of molecular and
condensed-phase
systems.\cite{Barbatti11_1759,Tavernelli14_62,Tully12_22A301,Tully98_407,Hynes14_97}
TSH is only tractable, however, if the electronic structure theory calculations
required at each step in the numerical integration of the molecular equations
of motion  may be reasonably afforded.  In an effort to enable simulation of
larger systems, there has been a push recently to utilize low-scaling, single
reference electronic structure methods in TSH dynamics
studies.\cite{Lan15_1360,Rothlisberger07_023001,Li16_935} While TSH with single
reference  methods is well-suited for describing non-radiative relaxation
within the excited state manifold\cite{Subotnik14_4253,Barbatti14_1395}, most
single reference  methods give a fundamentally flawed description of
non-radiative processes that terminate via internal conversion to the ground
electronic state.  Specifically, methods which represent the ground state by a
single Slater determinant and the excited states as linear combinations of
determinants give an incorrect topology for the ground and excited
electronic energy surfaces in regions of the nuclear configuration space where
they approach degeneracy.\cite{Massimo14_3074, Martinez06_1039,Li16_pp-TSH}
Symmetry-mandated intersections in the ground and excited states' potential
energy surfaces (PES's) can be absent with single reference  methods, so
qualitatively incorrect timescales and mechanisms for ground state recovery
processes can be predicted.  



\section{Future Directions: Relativistic Electronic Dynamics with Multi-Configurational Wave Functions}
\label{sec:Future}

\bibliography{Journal_Short_Name,ppSH,Li_Group_References}
\end{document}
