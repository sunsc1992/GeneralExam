\documentclass[12pt]{article}
\usepackage{achemso}
\usepackage[margin=1.0in]{geometry}
\usepackage[capitalize]{cleveref}
\usepackage{setspace}
\usepackage{amsmath}             % for equation typesetting
\usepackage{amssymb}             % for equation typesetting
\usepackage{mathrsfs} 
\usepackage{wasysym}             % for geometric shapes
\usepackage{color}               % for colored fonts
\usepackage{setspace}            % for 1.5 and double spacing
\usepackage{graphicx}            % main graphics package
\usepackage{wrapfig}             % allow text wrapping around figures
\doublespacing

\usepackage{appendix}
\crefname{figure}{Fig.}{Figs.}
\Crefname{figure}{Figure}{Figures}
\crefname{table}{Tab.}{Tabs.}
\Crefname{table}{Table}{Tables}
\crefname{equation}{Eq.}{Eqs.}
\Crefname{equation}{Equation}{Equations}
\crefname{section}{Sec.}{Secs.}
\Crefname{section}{Section}{Sections}
\crefname{appsec}{appendix}{appendices}
\Crefname{appsec}{Appendix}{Appendixes}

% Mathematical Shortcuts
\newcommand{\pfrac}[2]{\frac{\partial #1}{\partial #2}}   % partial derivative
\newcommand{\difrac}[2]{\frac{d #1}{d #2}}                % derivative
\newcommand{\bpar}[1]{\left( #1 \right)}                  % big parentheses
\newcommand{\bbra}[1]{\left[ #1 \right]}                  % big brackets
\newcommand{\bbar}[1]{\left| #1 \right|}                  % big bars
\newcommand{\bra}[1]{\left\langle #1 \right\vert}         % bra
\newcommand{\ket}[1]{\left\vert #1 \right\rangle}         % ket
\newcommand{\inner}[2]{\left\langle #1 \left\vert\right. #2 \right\rangle}            % bracket
\newcommand{\innerop}[3]{\left\langle #1 \left\vert #2 \right\vert #3 \right\rangle}  % operator matrix element
\newcommand{\innersub}[4]{\langle \bd{#1}_{#2}, \bd{#3}_{#4} \rangle}                 % bracket with subscripts
\newcommand{\half}{\frac{1}{2}}                           % 1/2
\newcommand{\powfrac}[3]{\bpar{\frac{#1}{#2}}^{#3}}       % fraction raised to a power
\newcommand{\ii}{\infty}                                  % infinity symbol
\newcommand{\tquad}{\quad\quad\quad}                      % triple-quad spacing
\renewcommand{\Im}{\text{Im}}                             % imaginary symbol
\renewcommand{\Re}{\text{Re}}                             % real symbol

\newcommand*\mycommand[1]{\texttt{\emph{#1}}}
\newcommand*\suchthat[0]{\text{ }\vert\text{ }}
\newcommand*\complexmatfield[1]{\mathbb{C}^{#1\text{x}#1}}
\newcommand*\speciallinear[2]{\mathrm{UnL}(#1,#2)}
\newcommand*\complexspeciallinear[1]{\speciallinear{#1}{\mathbb{C}}}
\newcommand*\generallinear[2]{\mathrm{GL}(#1,#2)}
\newcommand*\complexgenerallinear[1]{\generallinear{#1}{\mathbb{C}}}
\newcommand*\mat[1]{\boldsymbol{#1}}



\title{General Exam}
\date{October 19, 2016 \\ CHB 239}
\author{David Williams-Young\\ Department of Chemistry, University of Washington}


\begin{document}
\maketitle


\newpage
\section{Introduction}

Electronic and nuclear motion are fundamental to our understanding of chemical
and physical phenomena. From electronic reorganization upon photoinduced charge
transfer in the chromophores of solar cells to the ultrafast excitonic dynamics
in nanocrystalline materials, molecular dynamics lies at the heart of chemistry.
As such, to properly model these phenomena theoretically, we must often venture
into the time-domain to capture the physics necessary to completely understand
the problem at hand. As such, this task constitutes a primary focus of modern 
\emph{ab initio} quantum molecular theory.



%This work is organized into four sections.
This work is organized into three subsequent sections.
%\cref{sec:Chebyshev} details the work I've contributed to accelerating
%electronic dynamics through the Chebyshev expansion of the quantum propagator
%in real-time density functional theory.
\cref{sec:pp-X2C} focuses on my recent work of the extension of the
particle-particle Tamm-Dancoff approximation to relativistic Hamiltonians.
\cref{sec:pp-TSH} describes my contribtion to semi-classical non-adiabatic
molecular dynamics through the application of the particle-particle Tamm-Dancoff
approximation to the trajectory surface hopping method.  Finally,
\cref{sec:Future} outlines my plan for future development in the field of
electronic structure and dynamics in the relativistic treament of non-adiabatic
phenomena in molecular systems.

%\section{The Chebyshev Expansion of the Quantum Propagator}
%\label{sec:Chebyshev}

% Relativistic pp-TDA
\section{The Relativistic Particle-Particle Tamm-Dancoff Approximation}
\label{sec:pp-X2C}


% TSH with pp-TDA
\section{Trajectory Surface Hopping within the Particle-Particle Tamm-Dancoff Approximation}
\label{sec:pp-TSH}
Test\cite{Li16_pp-TSH}

% Future
\section{Future Directions: Relativistic Electronic Dynamics with Multi-Configurational Wave Functions}
\label{sec:Future}

\bibliography{Journal_Short_Name,ppSH,Li_Group_References,Chebyshev}
\end{document}
