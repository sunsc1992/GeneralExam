\section{Trajectory Surface Hopping}
\label{sec:TSH}

As described in the seminal works on TSH\cite{Tully98_407, Tully90_1061}, via
insertion of \cref{eq:AdiaExp} into \cref{eq:WaveEq}, one may derive an
approximate equation of motion for an electronic superposition state evolving
along a classical nuclear trajectory,
\begin{align}
  &i  \partial_t c_K(t) = H_{KJ}(t) c_J(t) \label{eq:EOMCs} \\
  &H_{KJ}(t) = \delta_{KJ}E_J(t) - i d_{KJ}^\xi (t) \partial_t R_{\xi}(t) \label{eq:HamiltonianElec} \\
  &d_{KJ}^\xi (t) = \innerop{\Phi_K(\vc{R}(t),t)}{\nabla^\xi}{\Phi_J(\vc{R}(t),t)} \label{eq:NAC}
  \quad .
\end{align}
The nuclear position time evolution is governed by Newtonian mechanics,
\begin{equation}
  -\nabla E_c(t) = \vc{m}\cdot \partial_t^2\vc{R}(t) \label{eq:Newton}
  \quad.
\end{equation}
Here, $\vc{m}$ collects the masses for each nucleus, and $d_{IJ}^\xi$ is the
rank-3 non--adiabatic coupling (NAC) tensor that provides a connection on the
electronic manifold.  $\xi$ is an arbitrary nuclear coordinate.  While $E_J$ in
\cref{eq:HamiltonianElec} represents an arbitrary energy eigenvalue of adiabat
$J$, $E_c(t)$ ($c$ indicating ``current") from \cref{eq:Newton} is the
electronic eigenenergy designated by the TSH algorithm to drive the nuclear
evolution at time $t$.  

The probability of switching which adiabt is contributing the forces for the
nuclear equation of motion throughout a nuclear time step, $\Delta t_N$, may be
given by
\begin{equation}
g_{cK}(t + \Delta t_N) = -2 \int_t^{t + \Delta t_N} 
  \frac{\mathrm{Re}(c_c(t') c^*_K(t'))d_{cK}^\xi (t') \partial_{t'}
  R_{\xi}(t')}{c_c(t') c^*_K(t')}\mathrm{d}t'
\end{equation}
Throughout the TSH simulation, $g_{cK}$ is evaluated for all considered $K$
during each nuclear time--step. It is then compared to a uniformly distributed
random number $\eta \in (0,1)$. Given that $g_{cK} > \eta$ a transition is
called for, otherwise the nuclear time evolution remains dependent on the
``current" adiabat. A more thorough review of the TSH algorithm may be found in
my previous collaborations\cite{DBWY16_JCTC935}.

The representation of the equations of motion in
\cref{eq:EOMCs,eq:HamiltonianElec,eq:NAC,eq:Newton} are general to any method
used to obtain the electronic adiabatic states. There have been many extensions
of this methods to a vast number of different electronic structure methods.
Recently, I have extended this method for use within the particle--particle
Tamm--Dancoff approximation for the description of the electronic
states\cite{DBWY16_Submitted1}, of which results are presented in
\cref{sec:pp-TSH}. Although TSH has been applied to study ISC using perturbative
approaches of accounting for SOC effects, I argue that this treatment is
inherently flawed as the SOC that is argued to be so negligible as to be treated
perturbatively is exactly the term that gives rise to the physical phenomena
that they observe. To properly treat SOC in TSH, one must account for the SOC in
a non--perturbative manner through an \emph{ab initio} treatment of some
relativistic Hamiltonian.

