\documentclass[usepdftitle=false,10pt]{beamer}
\usetheme{Dresden}
\usepackage{amsmath}
\usepackage{mathrsfs}
\usepackage{amssymb}
\usepackage{xcolor}

\usepackage{enumitem}
\newlist{mylist}{itemize}{2}
\setlist[mylist]{label=$\square$}


\usepackage{pifont}
\newcommand{\cmark}{\color{green} \ding{51}}%
\newcommand{\xmark}{\color{red} \ding{55}}%
\newcommand{\done}{\rlap{$\square$}{\raisebox{2pt}{\large\hspace{1pt}\cmark}}%
\hspace{-2.5pt}}
\newcommand{\wontfix}{\rlap{$\square$}{\large\hspace{1pt}\xmark}}

% Mathematical Shortcuts
\newcommand{\pfrac}[2]{\frac{\partial #1}{\partial #2}}   % partial derivative
\newcommand{\difrac}[2]{\frac{d #1}{d #2}}                % derivative
\newcommand{\bpar}[1]{\left( #1 \right)}                  % big parentheses
\newcommand{\bbra}[1]{\left[ #1 \right]}                  % big brackets
\newcommand{\bbar}[1]{\left| #1 \right|}                  % big bars
\newcommand{\bra}[1]{\left\langle #1 \right\vert}         % bra
\newcommand{\ket}[1]{\left\vert #1 \right\rangle}         % ket
\newcommand{\inner}[2]{\left\langle #1 \left\vert\right. #2 \right\rangle}            % bracket
\newcommand{\innerop}[3]{\left\langle #1 \left\vert #2 \right\vert #3 \right\rangle}  % operator matrix element
\newcommand{\innersub}[4]{\langle \bd{#1}_{#2}, \bd{#3}_{#4} \rangle}                 % bracket with subscripts
\newcommand{\half}{\frac{1}{2}}                           % 1/2
\newcommand{\powfrac}[3]{\bpar{\frac{#1}{#2}}^{#3}}       % fraction raised to a power
\newcommand{\ii}{\infty}                                  % infinity symbol
\newcommand{\tquad}{\quad\quad\quad}                      % triple-quad spacing
\renewcommand{\Im}{\text{Im}}                             % imaginary symbol
\renewcommand{\Re}{\text{Re}}                             % real symbol

\newcommand*\mycommand[1]{\texttt{\emph{#1}}}
\newcommand*\suchthat[0]{\text{ }\vert\text{ }}
\newcommand*\complexmatfield[1]{\mathbb{C}^{#1\text{x}#1}}
\newcommand*\speciallinear[2]{\mathrm{UnL}(#1,#2)}
\newcommand*\complexspeciallinear[1]{\speciallinear{#1}{\mathbb{C}}}
\newcommand*\generallinear[2]{\mathrm{GL}(#1,#2)}
\newcommand*\complexgenerallinear[1]{\generallinear{#1}{\mathbb{C}}}
\newcommand*\mat[1]{\boldsymbol{#1}}

\newcommand*\vc[1]{\boldsymbol{#1}}
\newcommand*\op[1]{\mathcal{#1}}

% Redo iint
\renewcommand*\iint[0]{\int\hspace{-8pt}\int}

\newcommand\blfootnote[1]{%
  \begingroup
  \renewcommand\thefootnote{}\footnote{#1}%
  \addtocounter{footnote}{-1}%
  \endgroup
}

\renewcommand*\footnoterule{}

\title{Modeling Spin--Forbidden Processes via Explicit Treatment of \emph{ab initio} Relativistic Effects in Non--Adiabatic Dynamics}
\date{Wednesday, October 19, 2016, 10:00 AM \\ CHB 239}
\author{David Williams-Young\\ Department of Chemistry, University of Washington\\
}

\begin{document}

% Title Page
\begin{frame}
  \titlepage
\end{frame}

% Talk outline
\begin{frame}
  \frametitle{Outline}
  \begin{itemize}
    \item[\ding{228}] Motivation: Spin--Forbidden Processes
    \item[\ding{228}] Solution: Relativistic Non--Adiabatic Molecular Dynamics
%    \item Trajectory Surface Hopping
%    \item[\ding{228}] Relativistic Theory Primer
%    \item The Configuration Interaction Method
    \item[\ding{228}] Previous Work
    \item[\ding{228}] Future Directions
  \end{itemize}
\end{frame}

% What is ISC
\begin{frame}
  \frametitle{Spin--Forbidden Processes: Intersystem Crossing (ISC)}

  \noindent
  Intersystem crossing is characterized by an electronic transition between
  quantum states of differing spin multiplicity.

  \begin{center}
  \includegraphics[width=0.85\textwidth]{ISC} 
  \end{center}
  \vspace{-1cm}
  \blfootnote{\tiny Wikipedia, Intersystem Crossing}
\end{frame}

% Motivating example: PHOLEDs
\begin{frame}
  \frametitle{{\bf Ph}osphorescent {\bf O}rganic {\bf L}ight {\bf E}mmitting
  {\bf D}iodes (PHOLED)}
\end{frame}

% Necessary Ingredients
\begin{frame}
  \frametitle{Quantum Mechanical Ingredients for the Treatment of ISC}
  \begin{center}
  \includegraphics[width=0.7\textwidth]{ISC} 
  \end{center}
  \vspace{-0.5cm}
  A proper quantum mechanical treatment of ISC should include:\\
  \begin{mylist}
    \item Electronic Non--Adiabaticity
    \item Correlated Treatment of Electrons
    \item Proper Treatment of Spin--Orbit Coupling
  \end{mylist}
\end{frame}

% Current state of the art (1)
\begin{frame}
  \frametitle{Current State of the Art \emph{ab initio} Methods for Studying 
  ISC: Trajectory Surface Hoppingi (TSH)}

  Quantum Mechanically, molecular time--dependence is governed by a Hamiltonian
  wave equation,

  \begin{equation*}
    \bpar{\op{H}_N(t) + \op{H}_{el}(t)} \ket{\Psi (t)} = i\partial_t \ket{\Psi(t)}
  \end{equation*}

  ~\\
  The molecular wave function, $\ket{\Psi(t)}$ formally decomposes into a product
  of electronic and nuclear wave functions

  \begin{equation*} 
    \ket{\Psi (t)} = \ket{\Phi(\vc{R}(t),t)}\otimes\ket{\Theta(t)} 
  \end{equation*} 

  ~\\
  In TSH, we say our nuclei evolve classically
  \begin{equation*}
    \inner{\vc{R}}{\Theta (t)} = \prod_{A = 0}^{N_\mathrm{atoms}} 
    \delta^3(\vc{R} - \vc{R}_A(t))
  \end{equation*}

  \vspace{-0.5cm}
  \blfootnote{\tiny Cederbaum, L.S.; \emph{JCP} \textbf{2008}, 128, 124101}
  \blfootnote{\tiny Adedi, A.; \emph{et. al.}; \emph{PRL} \textbf{2010}, 105, 123002}
\end{frame}

% Current state of the art (2)
\begin{frame}
  \frametitle{Current State of the Art \emph{ab initio} Methods for Studying 
  ISC: Trajectory Surface Hopping (TSH)}

  Working in the basis of adiabatic states,
  \begin{equation*}
    \op{H}(t) \ket{\Psi_I (t)} = E_I(t) \ket{\Psi_I (t)}
    \text{ }\Longrightarrow \text{ }
    \op{H}_{el}(t)\ket{\Phi_I(\vc{R}(t),t)} = 
      \mathcal{E}_I(t)\ket{\Phi_I(\vc{R}(t),t)}
  \end{equation*}
  
  ~\\
  Time--evolution of an electronic superposition state is given by
  \begin{align*}
    &i  \partial_t c_K(t) = H_{KJ}(t) c_J(t) \\
    &H_{KJ}(t) = \delta_{KJ}\mathcal{E}_J(t) - i d_{KJ}^\xi (t) \partial_t R_{\xi}(t) \\
    &d_{KJ}^\xi (t) = \innerop{\Phi_K(\vc{R}(t),t)}{\nabla^\xi}{\Phi_J(\vc{R}(t),t)}
  \end{align*}

  \blfootnote{\tiny Tully, J.C., \emph{et. al.}; \emph{JCP} \textbf{1971}, 55, 562}
  \blfootnote{\tiny Tully, J.C.; \emph{Faraday Discuss.} \textbf{1998}, 110, 407}
\end{frame}

% Current state of the art (3)
\begin{frame}
  \frametitle{Current State of the Art \emph{ab initio} Methods for Studying 
  ISC: Trajectory Surface Hopping (TSH)}

  Nuclear position time--evolution is governed by Newton's equations,

  \begin{equation*}
    -\nabla E_c(t) = \vc{m}\cdot \partial_t^2\vc{R}(t)
  \end{equation*}

  ~\\
  The probatility of ``hopping" from the current electronic adiabat to another
  through out the course of a nuclear time--step is given by

  \begin{equation*}
    g_{cK}(t + \Delta t_N) = -2 \int_t^{t + \Delta t_N} 
      \frac{\Re(c_c(t') c^*_K(t'))d_{cK}^\xi (t') \partial_{t'}
      R_{\xi}(t')}{c_c(t') c^*_K(t')}\mathrm{d}t'
  \end{equation*}
  \blfootnote{\tiny Tully, J.C., \emph{et. al.}; \emph{JCP} \textbf{1971}, 55, 562}
  \blfootnote{\tiny Tully, J.C.; \emph{Faraday Discuss.} \textbf{1998}, 110, 407}
\end{frame}

% Treatment of ISC within TSH (1) -- Yes to non--adiabaticity
\begin{frame}
  \frametitle{Treatment of ISC within Trajectory Surface Hopping}
  \begin{center}
  \includegraphics[width=0.7\textwidth]{ISC} 
  \end{center}
  \vspace{-0.5cm}
  TSH treatment of ISC: 
  \begin{mylist}
    \item[\done] Electronic Non--Adiabaticity
    \item Correlated Treatment of Electrons
    \item Proper Treatment of Spin--Orbit Coupling
  \end{mylist}
\end{frame}

% Configuration Interaction (1)
\begin{frame}
  \frametitle{The Configuration Interaction (CI) Method}

  Most implementations of TSH utilize the CI expansion for their description
  of electronic adiabats,
  \begin{align*}
  &\ket{\Phi^N_I (\vc{R}(t),t)} = \sum_{\alpha \in \mathcal{K}^N}  
    \ket{\psi_\alpha (\vc{R}(t),t)} d^I_\alpha(t)
  \\
  & \mathcal{K}^N \subset \left\lbrace \{ \phi_p \} \suchthat \{ \phi_p \} \subset
  \mathcal{R}_0^N \text{ and } \mathrm{card}(\{ \phi_p \}) = N \right\rbrace
  \end{align*}
  
  ~\\
  The expansion coefficients are obtained via diagonalizing the Hamiltonian
  in the basis of Slater determinants
  \begin{align*}
  &H_{\alpha\beta}(t) d_\beta^I(t) = d_\alpha^I(t) \mathcal{E}^N_I(t)
  \\
  &H_{\alpha\beta}(t) =
  \innerop{\psi_\alpha(\vc{R}(t),t)}{\mathcal{H}_{el}(t)}{\psi_\beta(\vc{R}(t),t)}
  \end{align*}
  \blfootnote{\tiny Szabo, A.; \emph{Modern Quantum Chemistry}}
\end{frame}

% Configuration Interaction (2)
\begin{frame}
  \frametitle{The Configuration Interaction (CI) Method}

  The CI expansion is especially simple for TSH as the required entities take on
  very simple forms

  \begin{align*}
    &d_{JK}^\xi = \sum_{\alpha,\beta \in \mathcal{K}^N} 
      \delta_{\alpha\beta} d_\alpha^{J*} \nabla^\xi d_\alpha^K +
      d_\alpha^{J*} d_\beta^K 
      \innerop{\psi_\alpha}{\nabla^\xi}{\psi_\beta} +
      \mathcal{M}_\mathrm{NAC} \\ \\
    &\nabla \mathcal{E}_J = \sum_{\alpha,\beta \in \mathcal{K}^N} 
      d_\alpha^{J*} d_\beta^J \nabla H_{\alpha\beta} +
      \mathcal{M}_E
  \end{align*}
  \blfootnote{\tiny Yamaguchi, Y.; \emph{et. al.}; \emph{Analytic Derivative Methods in Molecular Electronic Structure Theory: A New Dimension to Quantum Chemistry and its Applications to Spectroscopy}}
\end{frame}

% Treatment of ISC within TSH (2) -- Yes to Correlation 
\begin{frame}
  \frametitle{Treatment of ISC within Trajectory Surface Hopping}
  \begin{center}
  \includegraphics[width=0.7\textwidth]{ISC} 
  \end{center}
  \vspace{-0.5cm}
  TSH treatment of ISC: 
  \begin{mylist}
    \item[\done] Electronic Non--Adiabaticity
    \item[\done] Correlated Treatment of Electrons
    \item Proper Treatment of Spin--Orbit Coupling
  \end{mylist}
\end{frame}

% Relativistic Theory Primer (1)
\begin{frame}
  \frametitle{Relativistic Theory Primer: The (Quasi)--Relativistic Many--Body 
  Molecular Hamiltonian}

  The (quasi)-relativistic many--body molecular Hamiltonian in the absence of 
  an external perturbation may be written (in atomic units) as
  \begin{align*}
%   &\op{H} = \op{H}_N + \op{H}_{el} \\
    &\op{H}_N = \sum_{A=1}^{N_\mathrm{atoms}} \frac{\vc{P}^2_A}{2m_A} \qquad\qquad
     \op{H}^\mathrm{DC}_{el} = 
       \sum_{i=1}^{N_e} h^D(i) + \sum_{i < j}^{N_e} g^C(i,j) + V_{NN} \\
    \\
    & h^D(i) = 
      \begin{pmatrix}
        V_{ne}(\vc{r}_i) && c\text{ }\vc{\sigma} \cdot \vc{p}_i\\
	c\text{ }\vc{\sigma} \cdot \vc{p}_i && V_{ne}(\vc{r}_i) - c^2
      \end{pmatrix} \qquad 
      g^C(i,j) = \frac{1}{\vert \vc{r}_i - \vc{r}_j\vert} + O(c^{-2})
      \\ \\ 
    & V_{ne}(\vc{r}_i) = \sum_{A=1}^{N_\mathrm{atoms}}
                    \int_{\mathbb{R}^3}\mathrm{d}^3\vc{R}
		    \frac{\Gamma_A(\vc{R})}{\vert \vc{r}_i - \vc{R}\vert}
      \qquad V_{NN} = \sum_{A<B}^{N_\mathrm{atoms}}
                    \iint_{\mathbb{R}^3}\mathrm{d}^3\vc{R}\mathrm{d}^3\vc{R}'
		    \frac{\Gamma_A(\vc{R})\Gamma_B(\vc{R}')}
		         {\vert \vc{R} - \vc{R}'\vert}
  \end{align*}
  \blfootnote{\tiny Reiher, M.; \emph{et al}. \emph{Relativistic Quantum Chemistry}}
\end{frame}

% Relativisitc Theory Primer (2)
\begin{frame}
  \frametitle{Relativistic Theory Primer: The (Quasi)--Relativistic Many--Body 
  Molecular Hamiltonian}

  The Dirac--Coulomb Hamiltoninan acts on a 4 dimensional Hilbert space
  known as a bispinor field
  \begin{equation*}
    \op{H}_{el}^\mathrm{DC} \ket{\Phi} \quad \rightarrow \quad 
    \ket{\Phi} = \begin{pmatrix} \ket{\Phi_L} \\ \ket{\Phi_S} \end{pmatrix}
  \end{equation*}
  Of which each L/S component is itself a 2 dimension Hilbert space
  \begin{equation*}
    \ket{\Phi_{L/S}} =
      \begin{pmatrix} \vert\Phi_{L/S}^\alpha\rangle \\ \vert\Phi_{L/S}^\beta\rangle
      \end{pmatrix}
  \end{equation*}
  \blfootnote{\tiny Reiher, M.; \emph{et al}. \emph{Relativistic Quantum Chemistry}}

\end{frame}

% Relativistic Theory Primer (3)
\begin{frame}
  \frametitle{Realivitistic Theory Primer: Effective Two--Component Relativisitc
  Hamiltonian}

  There exists a unitary transformation operator $\op{U}$ such that
  \begin{equation*}
    \op{H}_{el}^\mathrm{DC} \mapsto 
    \begin{pmatrix} \op{H}_{el}^+ && 0 \\ 0 && \op{H}_{el}^- \end{pmatrix} 
    %\qquad
%    \begin{pmatrix} \ket{\Phi_L} \\ \ket{\Phi_S} \end{pmatrix}\mapsto
%    \begin{pmatrix} \ket{\Phi^{2c}} \\ 0 \end{pmatrix}
  \end{equation*}
  This folds the contribution from the small component into the large component 
  \begin{equation*}
    \begin{pmatrix} \ket{\Phi_L} \\ \ket{\Phi_S} \end{pmatrix}\mapsto
    \begin{pmatrix} \ket{\Phi^{2c}} \\ 0 \end{pmatrix}
  \end{equation*}
  \blfootnote{\tiny Reiher, M.; \emph{et al}. \emph{Relativistic Quantum Chemistry}}
\end{frame}

% Relativistic Theory Primer (4)
\begin{frame}
  \frametitle{Realivitistic Theory Primer: Effective Two--Component Relativisitc
  Hamiltonian}

  Expanding the transformed two--component Hamiltonian in the basis of single
  particle Fock states (\emph{molecular orbitals}), we obtain
  \begin{equation*}
    \op{H}_{el} = \sum_{pq}h^\mathrm{2C}_{pq} c_p^\dagger c_q +
    \frac{1}{2}\sum_{pqrs} g_{pqsr}c_p^\dagger c_q^\dagger c_r c_s
  \end{equation*}

  \begin{equation*}
    g_{pqsr} = \iint_{\mathbb{R}^3}\mathrm{d}^3\vc{r} \mathrm{d}^3\vc{r}'
      \frac{\phi^*_p(\vc{r})\phi^*_q(\vc{r}')\phi_s(\vc{r})\phi_r(\vc{r}')}
           {\vert \vc{r} - \vc{r}' \vert} \qquad 
    \phi_p(\vc{r}) = 
      \begin{pmatrix} \phi^\alpha_p(\vc{r}) \\ \phi^\beta_p(\vc{r}) \end{pmatrix}
  \end{equation*}
  \blfootnote{\tiny Reiher, M.; \emph{et al}. \emph{Relativistic Quantum Chemistry}}

\end{frame}

% SOC in TSH currently
\begin{frame}
  \frametitle{Phenomenological Addition of Spin--Orbit Coupling (SOC) in
  Trajectory Surface Hopping}

  Traditionally, SOC is added to nonrelativistic wave functions 
  \emph{a posteriori} in TSH through the SO part of the Dirac--Coulomb
  Hamiltonian
  \begin{equation*}
    h_\mathrm{DC}^\mathrm{SO} (i) = \frac{1}{2c^2} \sum_{A=1}^{N_\mathrm{atoms}}
      \int_{\mathbb{R}^3}\mathrm{d}^3\vc{R}
      \frac{\Gamma_A(\vc{R})
        (( \vc{r}_i - \vc{R} ) \times \vc{p}_i)\cdot \vc{s}_i}
	{\vert \vc{r}_i - \vc{R} \vert^3}
  \end{equation*}

  This yields an extraneous coupling in the electronic equations of motion

  \begin{equation*}
    H_{KJ} = \delta_{KJ}\mathcal{E}_J(t) +
    {\color{red}
      \innerop{\Phi_K(\vc{R}(t),t)}{\mathcal{H}_{SO}^{DC}}{\Phi_J(\vc{R}(t),t)}
    }
    - i d_{KJ}^\xi (t) \partial_t R_{\xi}(t) 
  \end{equation*}
  \blfootnote{\tiny Chang, A.H.H.; \emph{JCP} \textbf{1993} 99, 6824}
  \blfootnote{\tiny Daniel, C.; \emph{JCP} \textbf{1997} 106, 1421}
\end{frame}


% How we're going to solve it with relativistic surface hopping dynamics
\begin{frame}
  \frametitle{Solution: Relativistic Non--Adiabatic Molecular Dynamics}
\end{frame}


% Previous Work
\begin{frame}
  \frametitle{Previous Work towards Relativistic Non--Adiabatic Molecular Dynamics}
\end{frame}

% pp-TDA
\begin{frame}
  \frametitle{The Particle--Particle Tamm--Dancoff Approximation (pp--TDA)}
\end{frame}

% pp-X2C
\begin{frame}
  \frametitle{The Relativistic Particle--Particle Tamm-Dancoff Approximation 
  (X2C--pp--TDA)}
\end{frame}

\begin{frame}
  \frametitle{Trajectory Surface Hopping within the pp--TDA}
\end{frame}
\end{document}
