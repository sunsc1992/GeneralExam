\linespread{1.0}
\section{Relativistic Non--Adiabatic Molecular Dynamics with CI Wave Functions}
\mylinesp
\label{sec:Future}

Extension of existing methodologies using relativistic CI wave functions for use
in non--adiabatic dynamics is a challenging endeavor. Existing relativistic
treatments of CI wave function have been limited to single--point calculations,
and many of the ingredients needed for use with molecular dynamics have not yet
been developed.
{\bf As most of the computational machinery required to extend trajectory
surface hopping to relativistic CI wave functions, the majority of the proposed
research will be focused on developing the methods required to facilitate the
extension.}

Examining \cref{eq:EOMCs,eq:HamiltonianElec,eq:NAC,eq:Newton}, it is clear to
see that if one has access to geometric gradients ($\nabla E_c$) and NAC tensor
elements, one can perform TSH using that method, neither of which are readily
available for relativistic CI wave functions. As for the geometric gradients,
some methods have been proposed for other, more complicated relativistic
two--component methods\cite{Nakai16_JCTC2181,Cremer15_JCP214106} than the X2C
Hamiltonian in the mean--field description of the wave function, but the current
state of geometric gradients for the X2C is limited to scalar relativistic
treatment which neglect SOC\cite{Gauss11_JCP084114}. No attempts have been made
to evaluate geometric gradients in the two--component relativistic treatment of
CI wave functions. Although gradients and NAC tensor elements for
non--relativistic CI wave functions have existed for many years, extension to
two--component relativistic methods is non--trivial as they both involve
differentiation of the transformation operator, $\op{U}$.

%In the X2C treatment of CI wave functions, the expressions for the energy
%gradients and NAC tensor elements look nearly identical. The only differences
%between them are scaling of certain terms by energy differences, which are
%inconsequential to the difficulties of the X2C method. Here, I will only discuss
%the energy gradient, keeping in mind that a virtually identical procedure holds
%for the NAC tensor elements. The energy expression for the CI eave function is
%given by,
%\begin{align}
%&\mathcal{E}_I = h^\mathrm{X2C}_{pq}\gamma^I_{pq} + 
% (pq|rs) \Gamma^I_{pqrs} \\
%&\gamma^I_{pq} = 
%  \sum_{\alpha,\beta \in \mathcal{K}} 
%  \innerop{\psi_\alpha}{a_p^\dagger a_q}{\psi_\beta} d_\alpha^{I*} d_\beta^I
%  \qquad
%  \Gamma^I_{pqrs} = \frac{1}{2} \sum_{\alpha,\beta\in\mathcal{K}}
%  \innerop{\psi_\alpha}{a_p^\dagger a_q a_r^\dagger a_s - \delta_{qr}a_p^\dagger
%  a_s}{\psi_\beta} d_\alpha^{I*} d_\beta^I
%\end{align}
%Here, the time and parameter dependence has been dropped for brevity. The term
%of consequence is in $\vc{h}^\mathrm{X2C}$, which is the large component part of
%the two--component transformed one--electron Dirac Hamiltonian in the MO basis.
%As the energy is variational with respect to the expansion coefficients, the
%energy gradient only involves derivatives of the Hamiltonian
%\begin{align}
%&\nabla_\xi \mathcal{E}_I = \left(\nabla_\xi h^\mathrm{X2C}_{pq}\right)
%  \gamma^I_{pq} + 
% \left( \nabla_\xi (pq|rs) \right)\Gamma^I_{pqrs} \quad.
%\end{align}
%The derivative of the X2C core Hamiltonain may be written as
%\begin{equation}
%\nabla_\xi h^\mathrm{X2C}_{pq} = \nabla_\xi\int_{\mathbb{R}^3} \phi_p(\vc{r})
%  \mathcal{H}_+ \phi_q(\vc{r}) \text{
% }\mathrm{d}^3\vc{r} \quad,
%\end{equation}
%which involves derivatives of $\mathcal{H}_+$. Recalling the form of the
%two--component transformation, the derivatives of $\mathcal{H}_+$ are dependant
%on derivatives of $\mathcal{U}$,
%\begin{equation}
%\begin{pmatrix} \nabla_\xi \mathcal{H}_+ && 0_2 \\ 0_2 && \nabla_\xi
%\mathcal{H}_- \end{pmatrix} =
%2\mathcal{U}^\dagger\mathcal{H}_D\left(\nabla_\xi\mathcal{U}\right) +
%\mathcal{U}^\dagger \bpar{\nabla_\xi\mathcal{H}_D} \mathcal{U}\quad.
%\end{equation}
%While the gradients of the Dirac Hamiltonian have been known for some time, 

The gradients of $\mathcal{U}$ are dependant on the two--componant
transformation scheme. In general, the form of the derivative of $\mathcal{U}$
is non--trival due to the fact that it is typically obtained via an
eigendecomposition.\cite{Dyall07_book,Reiher15_book} There exist schemes to perform these types of
differentiations analytically, 
\cite{Nakai16_JCTC2181,Cremer15_JCP214106,Gauss11_JCP084114}
however they have not yet been extended to the
X2C method with the inclusion of SOC.  Once the deivatives of $\mathcal{U}$ are
obtained, assembly of the geometric derivatives and NAC tensor elements is
possible. {\bf Thus, one of the primary tasks in the proposed research is to
facilitate the efficient evaluation gradient and non--adiabatic coupling tensor
elements for X2C CI wave functions.}

Once TSH may be performed using X2C CI wave functions, there exist many possible
test cases to validate the accuracy and usefulness of the method. Currently, my
plan is to choose a relatively small, one-dimensional test case much analogous
to the one presented in \cref{sec:pp-TSH}. This condition is nicely met with the
photodissociation of HCo(CO)$_4$. It is well known that a key mehanistic point
of the photodissociation of  HCo(CO)$_4$ involves a rapid ($<$50ps) ISC process
from the first singlet to first triplet excited state. \cite{Daniel97_JCP1421,
Daniel94_JPC9823} This system has already been studied using a perturbative
treatment of SOC in CAS wave functions by Daniel \emph{et al.}
\cite{Daniel97_JCP1421} In their study,
they discovered that the ISC in this process may be modelled along a single
one--dimension reaction coordinate.  {\bf To validate this method, I will
attempt to recreate and improve upon existing studies of the photodissociation
of HCo(CO)$_4$. Such a study will allow me to asses the accuracy and usefulness
of trajectory surface hopping using relativistic CI wave functions.}

I beleive that the \emph{ab initio} treatment of SOC throughout a TSH simulation
will provide a marked improvement over the perturbative treatments that
currently exist in the literature. Futher, as previous studies have been limited
to model reaction coordinates, such a direct relativistic treatment of SOC will
provide a much needed generalization to current methodologies. As my proposed
treatment makes minimal assumptions about the underlything quantum mechanics of
the molecular system, it may be used as a general tool to solve current and
future problems in the field of realtivistic non--adiabatic dyanamics.
