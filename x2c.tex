\section{The Exact Two-Component Method}
\label{sec:X2C}

In the relativistic treatment of quantum molecular systems, the Hamiltonian
of interest is that of the Dirac Hamiltonian,
\begin{equation}
\op{H}_D = 
\begin{pmatrix}
  V \otimes \vc{I}_2 && c \text { } p^k \otimes \vc{\sigma}_k \\
  c \text { } p^k \otimes \vc{\sigma}_k && (V - 2mc^2) \otimes \vc{I}_2
\end{pmatrix} \quad ,
\label{eq:DiracHam}
\end{equation}
which describes the relativistic behavior of a single electron (fermion) in the
presence of a scalar potential $V$. Here, $V$ collects all scalar potential
terms, $\vec{p}$ is the momentum operator (acting as an internal vector
potential), $\vec{\vc{\sigma}}$ is a vector whose elements are the Pauli
matrices, and $\vc{I}_2$ is the 2x2 identity matrix. The implicit parametric
dependence on time has been dropped for brevity. The fact that this is a single
particle operator is quite consequential in the relativistic treatment of
molecules in that $\op{H}_D$ describes the mean-field quantum nature of a
single electron in the presence of the $N-1$ other electrons and the nuclei
(encapsulated in $V$). This is due to the fact that, in general, a Lorentz
invariant many--body Dirac equation does not exist if the Coulomb interaction is
taken to be the interaction potential between charges particles. In this work I
will not be explicitly taking into account relativistic retardation effects,
hence this treatment is not completely relativistic. This treatment does,
however, contain the chemically relevant relativistic effects such as
SOC and  scalar relativistic effects.
From the matrix representation of $\op{H}_D$, it is clear to see that it
acts upon a four--dimensional Hilbert space known as a bispinor field ,
$\ket{\Phi^{4c}}$, whose components are themselves two--dimensional,
\begin{equation}
\ket{\Phi^{4c}} = \begin{pmatrix}
 \ket{\Phi_L} \\ \ket{\Phi_S}
\end{pmatrix} \quad.
\end{equation}
Here, $\ket{\Phi_L}$ and $\ket{\Phi_S}$ are the so--called large and small
components of the bispinor respectively, and are both two--dimensional in the
spin manifold (i.e. spinors).

Although it is possible, in principle, to obtain the eigenstates of
\cref{eq:DiracHam} directly, it as often advantageous from a practical as well
as aesthetic perspective to transform the full four component relativistic
equations into a decoupled two component form. This is often convenient as the
resulting expressions closely resemble those found in non--relativistic
electronic structure theory and therefore allow the employment of standard
electronic structure methods with only minor modifications. In general, such a
transformation takes the form of a unitary operator, $\op{U}$, such that
\begin{equation}
\op{U}: 
\op{H}_D \mapsto \begin{pmatrix}
\op{H}_+ && \vc{0}_2 \\ \vc{0}_2 && \op{H}_- 
\end{pmatrix} \quad \text{ and } \quad
\ket{\Phi^{4c}} \mapsto \begin{pmatrix}
 \ket{\Phi^{2c}} \\ \vc{0}_2
\end{pmatrix} \quad.
\end{equation}
Thus making the transformed $\ket{\Phi^{2c}}$ an eigenstate of $\op{H}_+$.
The transformation under $\op{U}$, in effect, folds the contributions from
the small component of the wave function into the large components. In principle,
such a transformation is exact if a proper $U$ may be found. It is the case
that such an exact transformation is not practical to obtain due to the
effective many--body $V$, thus approximate decoupling schemes must be developed.
Several decoupling schemes have been explored in recent years, but in this work
I will only employ the use of the ``exact" two--component (X2C) method, for which
the exact details may be found elsewhere.

Using the X2C transformation method, we are able to obtain a total effecting
two--component Hamiltonian, which may be written in second quantized form,
\begin{equation}
\op{H}_\mathrm{X2C} = h_{pq}^\mathrm{X2C}c_p^\dagger c_q + 
  \frac{1}{2} \inner{pq}{rs} c_p^\dagger c_r^\dagger c_s c_q + V_\mathrm{NN}
  \label{eq:X2CHam}
\end{equation}
where $c_p$ and $c_p^\dagger$ are single spinor particle annihilation and
creation operators respectively. $\vc{h}^\mathrm{X2C}$ is the effective
relativistic core--Hamiltonian which includes both spin--orbit and scalar
relativistic effects, $\inner{\cdot}{\cdot}$, is the Coulomb operator expressed
in the basis of spinor MOs and $V_\mathrm{NN}$ is the nuclear--nuclear repulsion
energy. Expressing \cref{eq:X2CHam} in basis of a single Slater determinant and
expanding each spinor MO in the AO basis ($\lbrace \chi_\mu \rbrace$),
\begin{equation}
\phi_p ( \vc{r} ) = \sum_\mu \chi_\mu(\vc{r}) C_{\mu p}
\qquad
C_{\mu p} = \begin{pmatrix}
C_{\mu p}^\alpha \\ C_{\mu p}^\beta
\end{pmatrix}
\quad , \label{eq:AO2MO}
\end{equation}
we arrive at a relativistic two--component analogue of the Roothaan--Hall
equations familiar to traditional mean--field electronic structure theory,
\begin{align}
&\frac{1}{2}\left(
  F^S_{\mu\nu} \cdot \vc{I}_2 + F^k_{\mu\nu} \cdot \vc{\sigma_k}
\right) C_{\nu p} = (S_{\mu\nu}\cdot \vc{I}_2) C_{\nu p}
\epsilon_p\label{eq:Roothaan}
%\\
%\nonumber \\
%&F^S_{\mu\nu} = \left(
%  2\left( \mu\nu \vert \kappa\lambda \right) - 
%  \left(  \mu\lambda \vert \kappa\nu \right) 
%\right) P^S_{\lambda \kappa} \nonumber \\
%&F^k_{\mu\nu} = -\left(  \mu\lambda \vert \kappa\nu \right) P^k_{\lambda \kappa}\\
%\nonumber \\
%&\vc{P}^S = \frac{1}{2}\mathrm{Tr}_\sigma (\vc{P}\vc{I}_2) \nonumber \\
%&\vc{P}^k = \frac{1}{2}\mathrm{Tr}_\sigma (\vc{P}\vc{\sigma}_k)
%\label{eq:traceRel}\\
%&P_{\mu\nu} = C_{\mu i} C_{\nu i}^\dagger \nonumber
\end{align}
Here, $\vc{S}$ is the AO overlap matrix, 
%and $(\cdot \vert \cdot)$ is the
%Coulomb operator in the AO basis.
and $\{\epsilon_p\}$ is the set of MO
eigenenergies. $\vc{F}^S$,$\vc{F}^k$ and $\vc{P}^S,\vc{P}^k$ are the scalar and
vector parts of the spinor Fock and density matrices in the AO basis
respectively, 
%and are related to the full spinor operators in terms of the spin
%trace, $\mathrm{Tr}_\sigma$, relations of \cref{eq:traceRel}. 
Is important to note that $C_{\mu p} \in \mathbb{C}^2$ (where the $\alpha$ and
$\beta$ indicies represent inseparable spin-up and spin-down components of
$C_{\mu p}$), thus elements of the total spinor density (and thus the spinor
Fock) are 2x2 complex matrices.  \Cref{eq:Roothaan} may be solved
self--consistently to obtain the lowest energy (single) Slater determinantal
description of the many--body system.

The fact that one may write the X2C Hamiltonian in second quantized form makes it
suitable for application to post--self consistent field descriptions of
electronic correlation. Recently, I have extended the particle--particle
Tamm--Dancoff approximation to X2C optimized wave functions to study the fine
structure splittings in atomic and molecular systems, of which results are
presented in \cref{sec:pp-X2C}.


