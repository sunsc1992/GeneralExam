\linespread{1.0}
\section{The Relativistic Particle--Particle Tamm--Dancoff Approximation}
\linespread{1.5}
\label{sec:pp-X2C}

With the recent introduction of the particle-particle random phase and
Tamm-Dancoff approximations to \emph{ab initio} theory, routine queries of
traditionally difficult systems, such as diradicals and doubly excited states,
have been made possible. However, although a wealth of 
\begin{wraptable}{r}{0.45\textwidth}
  \vspace{-0.5cm}
  \caption{\footnotesize Calculated and reference\cite{NIST_ASD} excited-state fine structure
  splittings (in meV) of some atomic systems. The presence of a superscript
  ``$\circ$" in the term symbol denotes an odd state with respect to space
  inversion.}
 \vspace{-0.3cm}
 \label{tb:SingleEx}
 \centering
 \begin{tabular}{llll}
 System                    & Level                              & pp-TDA & Ref\\
 \hline
 \multirow{4}{*}{Mg}        & $^3$P$^\circ_{1}-^3$P$^\circ_{0}$  & 2.77  & 2.49 \\
                            & $^3$P$^\circ_{2}-^3$P$^\circ_{1}$  & 5.55  & 5.05 \\
                            & $^3$P$^\circ_{1}-^3$P$^\circ_{0}$  & 0.42  & 0.41 \\
                            & $^3$P$^\circ_{2}-^3$P$^\circ_{1}$  & 0.84  & 0.84 \\
 \hline
 \multirow{4}{*}{Al$^+$}    & $^3$P$^\circ_{1}-^3$P$^\circ_{0}$  & 9.13   & 7.55  \\
                            & $^3$P$^\circ_{2}-^3$P$^\circ_{1}$  & 18.40  & 15.36 \\
                            & $^3$P$^\circ_{1}-^3$P$^\circ_{0}$  & 2.17   & 1.73  \\
                            & $^3$P$^\circ_{2}-^3$P$^\circ_{1}$  & 4.50   & 3.65  \\
 \hline
 \multirow{4}{*}{Si$^{2+}$} & $^3$P$^\circ_{1}-^3$P$^\circ_{0}$  & 18.97  & 15.94 \\
                            & $^3$P$^\circ_{2}-^3$P$^\circ_{1}$  & 38.40  & 32.45 \\
                            & $^3$P$^\circ_{1}-^3$P$^\circ_{0}$  & 5.08   & 4.10  \\
                            & $^3$P$^\circ_{2}-^3$P$^\circ_{1}$  & 11.04  & 9.07  \\
 \hline
 \end{tabular}
 \vspace{-0.5cm}
\end{wraptable}
inquiry has been directed to investigating these methods, the previous
formulations have been restricted to spin-collinear systems, leaving the methods
incapable of treating non-collinearity and spin-orbit relativistic effects in
excited states.  My recent extension of the pp-TDA to two--component
relativistic Hamiltonians\cite{DBWY16_Accepted1} to study fine structure
splittings (FSS) has helped to fill that void in the literature.


The extension of the pp-TDA to the X2C Hamiltonian is more of a technical hurdle
than one of formalism. As \cref{eq:X2CHam} is second quantized, solution of
\cref{eq:ppTDAEig} only requires knowledge of a collection of MOs optimized in
the presence of relativsic effects,
$\mathcal{R}_0^{(N-2)}$. This may be achieved via the X2C-HF procedure to obtain
a mean-field single Slater determinant description of the $(N-2)$-electron wave
function.

The presence of spin-orbit couplings causes the total spin, $\vec{\op{S}}$, and
orbital, $\vec{\op{L}}$, angular momentum operators to no longer commute with
the Hamiltonian, therefore they no longer generate good quantum numbers for the
system.  Instead, the total angular momentum
$\vec{\op{J}}=\vec{\op{L}}+\vec{\op{S}}$ is the fundamental quantity that should
be considered when classifying the electronic states of the system.  A direct
consequence of spin-orbit couplings on the spectra of atoms is the lifting of
some of the degeneracies that would be expected in the ground or excited
electronic states.  In my previous work, we calculated the electronic adaibats
\begin{wraptable}{hr}{0.45\textwidth}
 \vspace{-0.5cm}
  \caption{\footnotesize Excited-state fine structure splittings (in meV).}
 \label{tb:DoubleEx}
 \centering
 \begin{tabular}{llll}
 System                    & Level & pp-TDA & Ref\cite{NIST_ASD,Krupenie72_423}\\
 \hline
  O$_2$ & $^3\Delta_3-{^3\Delta_2}$ & 20.58 & 18.09 \\ \hline
  \multirow{2}{*}{Al$^+$} & $^3$P$_1-^3$P$_0$ & 9.20 & 7.75 \\ 
  & $^3$P$_2-^3$P$_1$ & 17.93 & 15.03 \\  \hline
  \multirow{2}{*}{Si$^{2+}$} & $^3$P$_1-^3$P$_0$ & 19.46 & 16.55 \\ 
  & $^3$P$_2-^3$P$_1$ & 37.88 & 32.06 \\    \hline
 \end{tabular}
 \vspace{-0.5cm}
\end{wraptable}
using our developed X2C-pp-TDA method of selected atomic systems and compared he
obtained FSS with experimental reference values\cite{NIST_ASD} to asses the
accuracy of the method. The excitation energies within any varient of the pp-TDA
method may be written as
\begin{equation}
\omega_I = \omega'_I - \omega'_0 = \mathcal{E}_I^{N} - \mathcal{E}_0^N 
\quad I > 0.
\end{equation}
In this work, I only present the comparison of the X2C-pp-TDA method to
experimental reference for brevity. A full comparison to the analogous method of
X2C-CIS and X2C-TDHF may be found in the published work\cite{DBWY16_Accepted1}.

\Cref{tb:SingleEx} summarizes the FSS results for a series single excitations
for atomic systems. It is clear to see that the X2C-pp-TDA method performs quite
well with respect to the experimental data. A general trend may be observed in
that the X2C-pp-TDA consistently overestimates the splittings as the atomic
charge of the underlying nucleus increases. This effect is magnified in the low
energy transitions while it is less apparent in the higher energy transitions.
This is due to the fact that the frontier orbitals of the ($N-2$)-reference
being used become sub-optimal in the proper description of the $N$-electron
system due to a contraction in the presence of higher nuclear charge. This leads
to an unphysically small energetic separation between the frontier orbitals of
the $N$-electron system which causes increasing errors due to an unphysical
increase in mixing. This problem is less obvious in higher energy excitation
because the higher lying orbitals are not as affected.

\Cref{tb:DoubleEx} summarized the FSS results for a series of excitations that
are traditionally inaccesible though single--reference methods: double
excitations.  As the pp-TDA treats the excited states as double electron
additions, it has complete access to all double excitations out of the valence
orbitals. In addition, for the case of molecular oxygen, we are able to recover
the triplet ground and excited states with minimal spin--contamination. It is a
well known problem with relatvistic methods that the breaking of spin--symmetry
leads to considerable spin--contamination for states of non--singlet
multiplicity . This can lead to considerable noise in the predicted excitation
energies. As the FSS are on the order of meV, even a small amount of noise can
lead to difficulty in identifing the lifting of deneracies. in the pp-TDA, the
$(N-2)$ reference is a singlet, which allows us to recover the $N$ electron
electronic states with only minimal spin--contamination which enables simple
identification of the degeneracy lifted states. The FSS results for these
excitations are, again, in excellent agreement with experiment.

Overall, the X2C--pp-TDA method performs quite will in the evaluation of FSS in
molecular systems. However, becuase of the extreme restriction of the involved
configurations in $\mathcal{K}_\mathrm{pp-TDA}^N$, only select number of
chemically relevant states are accessable, namely only those excitations out of
the valence orbitals. While a considerable able of chemistry does occur in the
valence orbitals, many interesting transition metal phenomena in i.e. transition
metal complex, such as metal--ligand charge transfer involve contributions from
orbitals well below the Fermi level. As such, the pp-TDA is flawed in treating
such phenomena, while a general CAS treatment is able to capure them.
