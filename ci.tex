\section{The Configuration Interaction Method}
\label{sec:ci}

In the CI expansion, the exact electronic wave function of $N$ electrons may be
written as a linear combination of Slater determinants,
\begin{align}
&\ket{\Phi^N_I (\vc{R}(t),t)} = \sum_{\alpha \in \mathcal{K}^N}  
  \ket{\psi_\alpha (\vc{R}(t),t)} d^I_\alpha(t)
\label{eq:CIExp} \\
& \mathcal{K}^N = \left\lbrace \{ \phi_p \} \suchthat \{ \phi_p \} \subset
\mathcal{R}_0^N \text{ and } \mathrm{card}(\{ \phi_p \}) = N \right\rbrace
\quad ,
\end{align}
where $\left\lbrace\ket{\psi_\alpha}\right\rbrace$ is a set of Slater
determinants, and $\left\lbrace d^I_\alpha \right\rbrace$ is a set of normalized
expansion coefficients for a particular electronic adiabat $I$. $\mathcal{K}^N$ is
a set of $N$--electron configurations, upon which each of the indexed Slater
determinants is based. The elements of $\mathcal{K}^N$ are sets (configurations)
of $N$ \emph{molecular orbitals} (MO), $\phi_p$, taken from a reference set of MOs,
$\mathcal{R}^N_0$.  In principle, the above expansion is exact given that all
possible electronic configurations are included.  Given that that the
cardinality of $\mathcal{R}^N_0$ is $K$, there are a total of $\mathcal{C}(K,N)$
(read ``K choose N") possible determinants that would need to be included in the
expansion. Such an exhaustive expansion is called \emph{full configuration
interaction} (FCI) and is rendered impractical for the majority of systems due
to the rapid growth of $\mathcal{C}(K,N)$.  

In practice, we  must restrict which Slater determinants we include in
\cref{eq:CIExp}. As a result, a vast proportion of research in the field of
electronic structure theory has been centered on the proper way to restrict the
expansion in such a way that all of the relevant physics is maintained while
extraneous information may be discarded. Such methods as complete active space
(CAS), multi-reference configuration interaction (MRCI), density matrix
renormalization group (DMRG), particle--particle propagator methods , and
truncated CI methods may all be described, at their core, as algorithms to
provide such a filtering of the physically relevant determinants of the CI
expansion.  Once a restriction of \cref{eq:CIExp} has been made, one need only
diagonalize the Hamiltonian in the basis of these different Slater determinants,
\begin{align}
&H_{\alpha\beta}(t) d_\beta^I(t) = d_\alpha^I(t) \mathcal{E}^N_I(t)
\label{eq:CIEig}\\
&H_{\alpha\beta}(t) =
\innerop{\psi_\alpha(\vc{R}(t),t)}{\mathcal{H}_{el}(t)}{\psi_\beta(\vc{R}(t),t)}
\quad,
\end{align}
where $\mathcal{H}_{el}(t)$ is the electronic component of the total
time--depenendant Hamiltonian and $\mathcal{E}^N_I(t)$ is the $I$-th electronic
eigenenergie for an $N$-electron adiabat. 
Of these methods, only two: CAS and the particle-particle polarization
propagator methods, will be discussed in detail. 

Since its inception, CAS has been regarded as the industry standard for accurate
\emph{ab initio} descriptions of molecular systems. This is perhaps due to the
fact that it is the CI method that most closely resembles FCI while still
maintaining computation efficiency. In CAS methods, one performs FCI on a
restricted set of electronic configurations specified by an active space. The
reference orbitals of $\mathcal{R}^N_0$ are split into three disjoint sets: core 
($\mathcal{I}$), active ($\mathcal{A}$) and virtual ($\mathcal{V}$) orbitals.
The core orbitals are taken to always be occupied, the virtual always
unoccupied, and the active orbitals are given the freedom of varying occupancy.
Thus, the restricted collection of configurations, $\mathcal{K}_\mathrm{CAS}$,
may be written as
\begin{equation}
\mathcal{K}^N_\mathrm{CAS} = 
\left\lbrace \{ \phi_p \} \suchthat 
\{ \phi_p \} \subset \mathcal{R}^N_0 \setminus \mathcal{V} \text{ and } 
\mathcal{I} \subset \{ \phi_p \} \text{ and }
\mathrm{card}(\{ \phi_p \}) = N \right\rbrace
\quad. \label{eq:CASAnsatz}
\end{equation}
In fact, CAS reduces to FCI in the case that 
$\mathcal{I},\mathcal{V} = \varnothing$ and $\mathcal{A} = \mathcal{R}_0^N$.
%While diagonalization of \cref{eq:CIEig} ensures that the solutions are
%stationary with respect to the expansion coefficients, it does not ensure that
%the solution is stationary with respect to the choice of $\mathcal{R}_0$. One
%way to remedy this in the case that more than the ground state is desires from
%\cref{eq:CIEig} is to introduce the state--averaged CAS self--consistent field
%(SA-CASSCF) method which ensures this condition.
%The practical implementation of the SA-CASSCF method are outside of
The practical implementation of the CAS method are outside of the scope of the
proposed work, and may be found in the seminal works on the method.


Recently, the particle-particle Tamm-Dancoff approximation (pp-TDA), which has
been a standard trade tool of the nuclear physics community in the treatment of
the many--body correlation energy for low matter density
systems\cite{SchuckBook_04}, have been extended to the treatment the correlation
energy and excitation energies of quantum molecular systems within a gaussian 
basis set.\cite{Yang13_224105,Yang13_18A522,
Yang13_174110,Yang13_104112,Yang13_030501,Yang09_066403,Bulik13_104113}.
%Although this introduction into the quantum chemistry community in relatively
%recent, a wealth of effort has been afforded to the rigorous investigation of
%these methods in a variety of different contexts, involving the evaluation of
%excitation energies\cite{Yang13_224105,Yang13_18A522,Yang13_174110}.
Although it is not standard to do so, it is possible to formulate the pp-TDA as
a special case of CAS with a very restricted active space.

In the pp-TDA, the $N$-electron ground and excited states may be written as two
particle additions to an $(N-2)$-electron reference determinant,
\begin{equation}
\ket{\Phi_I^N (\vc{R}(t),t)} = \sum_{a < b} X_{ab}^I(\vc{R}(t),t)
a_a^\dagger(\vc{R}(t)) a_b^\dagger (\vc{R}(t))
\ket{\psi_0^{(N-2)}(\vc{R}(t),t)} \quad, \label{eq:ppTDAState}
\end{equation}
where $a_a^\dagger$ is a single particle creation operator into the the $a$-th
MO and $X_{ab}^I$ is an expansion coefficient corresponding to the addition of
two electrons into the $(N-2)$--electron Slater determinant. Given that
$\ket{\psi_0^{(N-2)}}$ is constructed with a particular electronic
configuration, $\{ \phi_p \}^{(N-2)}_0 \subset \mathcal{R}^{(N-2)}_0$, we may recast
\cref{eq:ppTDAState} as a linear combintation of electronic configurations that
is exactly a specialization of \cref{eq:CASAnsatz} with
$\mathcal{V} = \varnothing$ and $\mathcal{I} = \{ \phi_p \}^{(N-2)}_0$,
\begin{equation}
\mathcal{K}_\mathrm{pp-TDA}^N =
\left\lbrace \{ \phi_p \}^{(N-2)}_0 \cup \{\phi_a,\phi_b \} \suchthat 
\{ \phi_a,\phi_b \} \subset \mathcal{R}^{(N-2)}_0 \setminus \{\phi_p \}^{(N-2)}_0 
\right\rbrace \quad.
\end{equation}
Using the Slater--Condon rules for the evaluation of matrix elements between
Slater determinants\cite{Szabo}, we may write \cref{eq:CIEig} as an eigenproblem
in terms of the expansion coefficients in \cref{eq:ppTDAState},
\begin{align}
&\sum_{c < d} A_{ab,cd} X_{cd}^I = \omega'_I X_{ab}^I \quad (a < b)\\
&A_{ab,cd} = \delta_{ac}\delta_{bd}(\varepsilon_a + \varepsilon_b) + 
  \innerop{ab}{}{cd} \\
&\omega'_I = \mathcal{E}^{N}_I - \mathcal{E}^{(N-2)}_0
\quad.
\end{align}
Here, $\{\varepsilon_p\}$ is the set of MO eigenenergies,
$\innerop{\cdot}{}{\cdot}$ is an antisymmetrized electron repulsion integral,
and $\mathcal{E}_0^{(N-2)}$ is the electronic energy of $\ket{\psi_0^{(N-2)}}$,
\begin{equation}
\mathcal{H}_{el}\ket{\psi_0^{(N-2)}} = \mathcal{E}_0^{(N-2)}\ket{\psi_0^{(N-2)}}
\quad.
\end{equation}

The pp-TDA is the basis for much of my previous work which will be discussed in
\cref{sec:pp-X2C,sec:pp-TSH}, as it provides a simpler route to the corrlelated
treatment of electronic adiabatic states over a full CAS treatment. As will be
discussed, it does have limitations which can be resolved by a general CAS
treatment. As such, CAS will be the primary focus of my future work in this
field (\cref{sec:Future}).
