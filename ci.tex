\section{The Configuration Interaction Method}
\label{sec:ci}

In the CI expansion, the exact electronic wave function of $N$ electrons may be
written as a linear combination of Slater determinants,
\begin{align}
&\ket{\Phi_I (\vc{R}(t),t)} = \sum_{\alpha \in \mathcal{K}}  
  \ket{\psi_\alpha (\vc{R}(t),t)} d^I_\alpha(t)
\label{eq:CIExp} \\
& \mathcal{K} = \left\lbrace \{ \phi_p \} \suchthat \{ \phi_p \} \subset
\mathcal{R}_0 \text{ and } \mathrm{card}(\{ \phi_p \}) = N \right\rbrace
\quad ,
\end{align}
where $\left\lbrace\ket{\psi_\alpha}\right\rbrace$ is a set of Slater
determinants, and $\left\lbrace d^I_\alpha \right\rbrace$ is a set of normalized
expansion coefficients for a particular electronic adiabat. $\mathcal{K}$ is a
set of electronic configurations, upon which each of the indexed Slater
determinants is based. The elements of $\mathcal{K}$ are sets (configurations)
of \emph{molecular orbitals} (MO), $\phi_p$, taken from a reference set of MOs,
$\mathcal{R}_0$.  In principle, the above expansion is exact given that all
possible electronic configurations are included.  Given that that the
cardinality of $\mathcal{R}_0$ is $K$, there are a total of $\mathcal{C}(K,N)$
(read ``K choose N") possible determinants that would need to be included in the
expansion. Such an exhaustive expansion is called \emph{full configuration
interaction} (FCI) and is rendered impractical for the majority of systems due
to the rapid growth of $\mathcal{C}(K,N)$.  In practice, we  must restrict which
Slater determinants we include in \cref{eq:CIExp}. As a result, a vast
proportion of research in the field of electronic structure theory has been
centered on the proper way to restrict the expansion in such a way that all of
the relevant physics is maintained while extraneous information may be
discarded. Such methods as complete active space (CAS), multi-reference
configuration interaction (MRCI), density matrix renormalization group (DMRG),
particle--particle propagator methods , and truncated CI methods may all be
described, at their core, as algorithms to provide such a filtering of the
physically relevant determinants of the CI expansion.  Once a restriction of
\cref{eq:CIExp} has been made, one need only diagonalize the Hamiltonian in the
basis of these different Slater determinants,
\begin{align}
&H_{\alpha\beta}(t) d_\beta^I(t) = d_\alpha^I(t) \mathcal{E}_I(t)\\
&H_{\alpha\beta}(t) =
\innerop{\phi_\alpha(\vc{R}(t),t)}{\mathcal{H}_{el}(t)}{\phi_\beta(\vc{R}(t),t)}
\quad,
\end{align}
where $\mathcal{H}_{el}(t)$ is the electronic component of the total
time--depenendant Hamiltonian and $\mathcal{E}_I(t)$ is the $I$-th electronic
eigenenergie. 
Of these methods, only two: CAS and the particle-particle polarization
propagator methods, will be discussed in detail. 
