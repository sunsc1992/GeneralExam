\documentclass[12pt]{article}
\usepackage{achemso}
\usepackage[margin=1.0in]{geometry}
\usepackage{setspace}
\usepackage{amsmath}             % for equation typesetting
\usepackage{amssymb}             % for equation typesetting
\usepackage{mathrsfs} 
\usepackage{wasysym}             % for geometric shapes
\usepackage{color}               % for colored fonts
\usepackage{setspace}            % for 1.5 and double spacing
\usepackage{graphicx}            % main graphics package
\usepackage{wrapfig}             % allow text wrapping around figures
\usepackage{subcaption}
%\doublespacing
\linespread{1.5}

\usepackage{appendix}

\usepackage[capitalize]{cleveref}
\crefname{figure}{Fig.}{Figs.}
\Crefname{figure}{Figure}{Figures}
\crefname{table}{Tab.}{Tabs.}
\Crefname{table}{Table}{Tables}
\crefname{equation}{Eq.}{Eqs.}
\Crefname{equation}{Equation}{Equations}
\crefname{section}{Sec.}{Secs.}
\Crefname{section}{Section}{Sections}
\crefname{appsec}{appendix}{appendices}
\Crefname{appsec}{Appendix}{Appendixes}

%\usepackage{titlesec}
%\titleformat*{\section}{\Large\bfseries}
%\titleformat*{\subsection}{\large\bfseries}

% Mathematical Shortcuts
\newcommand{\pfrac}[2]{\frac{\partial #1}{\partial #2}}   % partial derivative
\newcommand{\difrac}[2]{\frac{d #1}{d #2}}                % derivative
\newcommand{\bpar}[1]{\left( #1 \right)}                  % big parentheses
\newcommand{\bbra}[1]{\left[ #1 \right]}                  % big brackets
\newcommand{\bbar}[1]{\left| #1 \right|}                  % big bars
\newcommand{\bra}[1]{\left\langle #1 \right\vert}         % bra
\newcommand{\ket}[1]{\left\vert #1 \right\rangle}         % ket
\newcommand{\inner}[2]{\left\langle #1 \left\vert\right. #2 \right\rangle}            % bracket
\newcommand{\innerop}[3]{\left\langle #1 \left\vert #2 \right\vert #3 \right\rangle}  % operator matrix element
\newcommand{\innersub}[4]{\langle \bd{#1}_{#2}, \bd{#3}_{#4} \rangle}                 % bracket with subscripts
\newcommand{\half}{\frac{1}{2}}                           % 1/2
\newcommand{\powfrac}[3]{\bpar{\frac{#1}{#2}}^{#3}}       % fraction raised to a power
\newcommand{\ii}{\infty}                                  % infinity symbol
\newcommand{\tquad}{\quad\quad\quad}                      % triple-quad spacing
\renewcommand{\Im}{\text{Im}}                             % imaginary symbol
\renewcommand{\Re}{\text{Re}}                             % real symbol

\newcommand*\mycommand[1]{\texttt{\emph{#1}}}
\newcommand*\suchthat[0]{\text{ }\vert\text{ }}
\newcommand*\complexmatfield[1]{\mathbb{C}^{#1\text{x}#1}}
\newcommand*\speciallinear[2]{\mathrm{UnL}(#1,#2)}
\newcommand*\complexspeciallinear[1]{\speciallinear{#1}{\mathbb{C}}}
\newcommand*\generallinear[2]{\mathrm{GL}(#1,#2)}
\newcommand*\complexgenerallinear[1]{\generallinear{#1}{\mathbb{C}}}
\newcommand*\mat[1]{\boldsymbol{#1}}

\newcommand*\vc[1]{\boldsymbol{#1}}
\newcommand*\op[1]{\mathcal{#1}}


\title{General Exam}
\date{October 19, 2016 \\ CHB 239}
\author{David Williams-Young\\ Department of Chemistry, University of Washington}


\begin{document}
\linespread{1.0}
\maketitle
\linespread{1.5}

\newpage
\section{Introduction}

Formally spin--forbidden processes, such as intersystem crossing (ISC),  play
an crucial role in photochemistry. ISC is a transition between quantum states
of differing spin multiplicity which manifests either radiatively or
non-radiatively.  The functionality of many modern technologies depend heavily
on finely tuned ISC to control the rate of population and depopulation of
triplet intermediate states from singlet states upon photoexcitation. The fact
that the non--relativistic treatment of molecular quantum mechanics predicts
phenomena such as ISC as formally forbidden, it has been long thought that the
time--scales at which they occur are far too long to compete with transitions
between states of the same spin multiplicity. However, recent studies have
shown that this is not necessarily the case, and that ISC can indeed be
competitive even at short time--scales.  ISC is an inherently relativistic, and
non--adiabatic time--dependent process. Thus to accurately treat these phenomena
theoretically, one must provide an \emph{ab initio} description of the
relativistic effects throughout the non--adiabatic time--evolution of the
quantum system.  {\bf While much effort has been directed at the accurate
treatment of non--adiabatic molecular dynamics, the scope of inquiry has been
primary limited to non--ralativistic regime.}

Relativistic effects, while often neglected in most standard treatments of
quantum mechanics, can have profound consequences in chemical
systems.\cite{Pyykko12_45} Scalar relativistic effects cause the contraction of
the core electron shells of heavy atoms, but perhaps of even more consequence is
the introduction of spin couplings in the Hamiltonian.  Spin-spin (SSC) and
spin-orbit (SOC) coupling can affect the electronic spin dynamics even in light
atoms. A direct consequence of these couplings on the electronic manifold is the
breaking of spin--symmetry as the Hamiltonian no longer commutes with the spin
operator, $\op{S}$. This breaking of spin--symmetry is what allows, at an
operator level, for formally spin--forbidden processes to occur, namely
ISC. Although some approaches have been purposed to account for SOC in the
treatment of the electronic manifold perturbatively\cite{Thiel14_JCP124101},
these approximations will break down whenever SOC is non--negligible. Hence a
variational, and well balanced approach much be adopted to account for these
interactions for the general case. There exist several \emph{ab initio} methods
to approximately treat relativisic effects, however, in this work the discussion
will be limited to two--component relativistic Hamiltonians, specificailly the
``exact two--component" (X2C) Hamiltonian. A breif overview of the X2C method
may be found in \cref{sec:X2C}.

In the quantum mechanical description of molecular systems, neglecting explicit
coupling to a quantized photon field, time evolution of the total molecular wave
function, $\ket{\Psi (t)}$ is governed by the Hamiltonian wave equation 
(in atomic units),
\begin{equation}
\op{H}(t) \ket{\Psi (t)} = i\partial_t \ket{\Psi(t)} \quad,
\label{eq:WaveEq}
\end{equation}
where $\op{H}(t)$ is the time--dependent Hamiltonian, $\partial_t$ is a partial
derivative with respect to time, and $t$ is a time parameter.  The moieties
enclosed in parentheses are taken to be parameters, i.e. the wave function is
parameterized by time.  In principle, one may solve (in some approximate manner)
\cref{eq:WaveEq} simultaneously for both the electronic and nuclear degrees of
freedom explicitly. This approach is, however, intractable for quantum systems
exceeding more than a few particles. To simplify the solutions of
\cref{eq:WaveEq}, one may formally decompose $\ket{\Psi (t)}$ into a product of
nuclear and electronic wave functions,
\begin{equation} 
\ket{\Psi (t)} = \ket{\Phi(\vc{R}(t),t)}\otimes\ket{\Theta(t)} 
\quad .  
\label{eq:exactSepElecNuc}
\end{equation} 
Here, $\ket{\Phi(\vc{R}(t),t)}$ and $\ket{\Theta (t)}$ are the electronic and
nuclear wave functions respectively and $\vc{R}(t)$ is the expectation value of
the nuclear position operator, $\vc{R}(t) =
\innerop{\Theta(t)}{\hat{\vc{R}}}{\Theta(t)}$.  
This formally exact treatment\cite{Gross10_PRL123002, Cederbaum08_JCP124101,
Ghosh15_MP1} allows one to, at least in in principle, bifurcate the treatment of
quantum molecular dynamics into explicit treatment of the electronic and nuclear
time dependence separately.  Thus, in order to theoretically treat molecular
dynamics accurately, one must be able to treat both the electronic and nuclear
degrees of freedom to some level of accuracy.

When obtaining the time evolution of the total molecular wave function, it is
often advantageous to work in so called adiabatic basis (the eigenstates of
\cref{eq:WaveEq}) of quantum states,
\begin{equation}
\op{H}(t) \ket{\Psi_I (t)} = E_I(t) \ket{\Psi_I (t)}
\quad \forall t, I \in \mathbb{N},
\label{eq:EigSpec}
\end{equation}
where $\ket{\Psi_I}$ and $E_I$ represent the $I$-th adiabatic wave function and
eigenenergie, respectively. For the purposes of the current discussion, we will
be assuming a discrete eigenspectrum of the Hamiltonian that is smooth over $t$.
Taking the separation of the total wave function in \cref{eq:exactSepElecNuc},
we may rewrite the total wave function as a linear combination of adiabatic
states,
\begin{equation}
\ket{\Psi (t)} = c_I \ket{\Phi_I (\vc{R}(t),t)} \otimes \ket{\Theta_I (t)}
\quad ,
\label{eq:AdiaExp}
\end{equation}
where $\{ c_I \}$ is a set of expansion coefficients, and $\{\ket{\Phi_I}\}$ and
$\{\ket{\Theta_I}\}$ are the sets of electronic and nuclear adiabatic wave
functions, respectively. Given a complete eigenspectrum of \cref{eq:EigSpec},
the expansion in \cref{eq:AdiaExp} is formally exact, thus the problem now
becomes how to properly obtain the adiabatic electronic and nuclear wave
functions.

There exist a vast plethora of computational methods to treat molecular dynamics
based on the separation in \cref{eq:exactSepElecNuc,eq:AdiaExp}. For the
purposes of this work, the discussion will be restricted to that of the
trajectory surface hopping (TSH) method.  TSH is one of the most widely applied
methods for simulating electronically non-adiabatic dynamics of molecular and
condensed-phase systems.\cite{Barbatti11_1759, Tavernelli14_62, Tully12_22A301,
Tully98_407, Hynes14_97} At its core, TSH is a stochastic algorithm that
controls which electronic adiabat dictates the forces on the nuclei during a
classical nuclear trajecory.\cite{Preston71_562} As a semi-classical method, the
time--evolution of the electronic degrees of freedom are treated as quantum
mechanical while the time evolution of the nuclear positions are treated as
classical through Newton's equations of motion. In essence, this classical
treatment may be thought of as the nuclear wave function begin written as a
product of delta functions centered at the classical nuclear positions
\begin{equation}
\inner{\vc{R}}{\Theta_I (t)} = 
  \prod_{A = 0}^{N_\mathrm{atoms}} \delta^3(\vc{R} - \vc{R}_A(t))
  \quad \forall I,
  \label{eq:ClassicalNuclei}
\end{equation}
where $N_\mathrm{atoms}$ is the number of nuclei in the molecule.  This is often
a valid approach, as the nuclei of molecules are orders of magnitude heavier
than the electrons, and thus move much slower. TSH provides a somewhat ideal
altorithm for the simulation and interpretation of many non--adiabatic
processes, such as ISC, as it is somewhat blind to the underlying method used to
obtain the electronic adiabats. Thus it is general to cases where the coupling
between electronic adiabats is large or small, all while allowing smooth
transition between the two. A brief overview of the TSH algorithm may be found
in \cref{sec:TSH}. While some attempts have been made to study ISC using TSH via
a perturbative inclusion of SOC in the electronic adiabats
\cite{Thiel14_JCP124101}, no attempts have been made to provide an explcitly
relativistic treatment of these effects within TSH. These perturbative
approaches are valid only in the limit of negligible SOC, which seems to
contradict the premise as it is in the SOC that ISC is made possible. Hence this
lack of relativisitic treatment of SOC in TSH is an inherient flaw in current
methodologies to model ISC and indicates a need for the development of new
theoretical methods to model the electronic adiabatic states realtivistically.

The quality of TSH simulations is heavily dependent on the quality of the
treatment of the electronic adibatic states.  Over the years, perhaps the most
successful \emph{ab initio} treatment of the adiabatic electronic wave function
has been in the configuration interaction (CI) expansion of the electronic wave
function. In the CI expansion, the exact electronic wave function of $N$
electrons may be written as a linear combination of Slater determinants,
\begin{align}
&\ket{\Phi (\vc{R}(t),t)} = \sum_{\alpha \in \mathcal{K}}  
  \ket{\psi_\alpha (\vc{R}(t),t)} d_\alpha(t)
\label{eq:CIExp} \\
& \mathcal{K} = \left\lbrace \{ \phi_p \} \suchthat \{ \phi_p \} \subset
\mathcal{R}_0 \text{ and } \mathrm{card}(\{ \phi_p \}) = N \right\rbrace
\quad ,
\end{align}
where $\left\lbrace\ket{\psi_\alpha}\right\rbrace$ is a set of Slater
determinants, and $\left\lbrace d_\alpha \right\rbrace$ is a set of normalized
expansion coefficients. $\mathcal{K}$ is a set of electronic configurations,
upon which each of the indexed Slater determinants is based. The elements of
$\mathcal{K}$ are sets (configurations) of \emph{molecular orbitals} (MO),
$\phi_p$, taken from a reference set of MOs, $\mathcal{R}_0$.  In principle, the
above expansion is exact given that all possible electronic configurations are
included.  Given that that the cardinality of $\mathcal{R}_0$ is $K$, there are
a total of $\mathcal{C}(K,N)$ (read ``K choose N") possible determinants that
would need to be included in the expansion. Such an exhaustive expansion is
called \emph{full configuration interaction} (FCI) and is rendered impractical
for the majority of systems due to the rapid growth of $\mathcal{C}(K,N)$.
In practice, we  must restrict which Slater determinants we include in
\cref{eq:CIExp}. As a result, a vast proportion of research in the field of
electronic structure theory has been centered on the proper way to restrict the
expansion in such a way that all of the relevant physics is maintained while
extraneous information may be discarded. Such methods as complete active space
(CAS), multireference configuration interaction (MRCI), density matrix
renormalization group (DMRG), particle--particle propagator methods , and
truncated CI methods may all be described, at their core, as algorithms to
provide such a filtering of the physically relevant determinants of the CI
expansion. 

Application of CI varients to TSH is well established in the literature.
Recently, I have extended TSH for use within the particle--particle
Tamm--Dancoff approximation (pp-TDA) for the description of the electronic
states\cite{DBWY16_Submitted1}, of which results are presented in
\cref{sec:pp-TSH}. Although TSH has been applied to study ISC using perturbative
approaches of accounting for SOC effects, I argue that this treatment is
inherently flawed as the SOC that is argued to be so negligible as to be treated
perturbatively is exactly the term that gives rise to the physical phenomena
that they observe. To properly treat SOC in TSH, one must account for the SOC in
a non--perturbative manner through an \emph{ab initio} treatment of some
relativistic Hamiltonian.

\section{Trajectory Sufrace Hopping}
\label{sec:TSH}

As described in the seminal works on TSH\cite{Tully98_407, Tully90_1061}, via
insertion of \cref{eq:AdiaExp} into \cref{eq:WaveEq}, one may derive an
approximate equation of motion for an electronic superposition state evolving
along a classical nuclear trajectory,
\begin{align}
  &i  \partial_t c_K(t) = H_{KJ}(t) c_J(t) \label{eq:EOMCs} \\
  &H_{KJ}(t) = \delta_{KJ}E_J(t) - i d_{KJ}^\xi (t) \partial_t R_{\xi}(t) \label{eq:HamiltonianElec} \\
  &d_{KJ}^\xi (t) = \innerop{\Phi_K(\vc{R}(t),t)}{\nabla^\xi}{\Phi_J(\vc{R}(t),t)} \label{eq:NAC}
  \quad .
\end{align}
The nuclear position time evolution is governed by Newtonian mechanics,
\begin{equation}
  -\nabla E_c(t) = \vc{m}\cdot \partial_t^2\vc{R}(t) \label{eq:Newton}
  \quad.
\end{equation}
Here, $\vc{m}$ collects the masses for each nucleus, and $d_{IJ}^\xi$ is the
rank-3 non--adiabatic coupling (NAC) tensor that provides an affine topological
connection on the electronic manifold through the nuclear momentum operator.
$\xi$ is an arbitrary nuclear coordinate.  While $E_J$ in
\cref{eq:HamiltonianElec} represents an arbitrary energy eigenvalue of adiabat
$J$, $E_c(t)$ ($c$ indicating ``current") from \cref{eq:Newton} is the
electronic eigenenergy designated by the TSH algorithm to drive the nuclear
evolution at time $t$.  

%Talk about algorithm to switch the state throughout the dynamics


\section{The Exact Two-Component Method}
\label{sec:X2C}

In the relativistic treatment of quantum molecular systems, the Hamiltonian
of interest is that of the Dirac Hamiltonian,
\begin{equation}
\op{H}_D = 
\begin{pmatrix}
  V \otimes \vc{I}_2 && c \text { } p^k \otimes \vc{\sigma}_k \\
  c \text { } p^k \otimes \vc{\sigma}_k && (V - 2mc^2) \otimes \vc{I}_2
\end{pmatrix} \quad ,
\label{eq:DiracHam}
\end{equation}
which describes the relativistic behavior of a single electron (fermion) in the
presence of a scalar potential $V$. Here, $V$ collects all scalar potential
terms, $\vec{p}$ is the momentum operator (acting as an internal vector
potential), $\vec{\vc{\sigma}}$ is a vector whose elements are the Pauli
matrices, and $\vc{I}_2$ is the 2x2 identity matrix. The implicit parametric
dependence on time has been dropped for brevity. The fact that this is a single
particle operator is quite consequential in the relativistic treatment of
molecules in that $\op{H}_D$ describes the mean-field quantum nature of a
single electron in the presence of the $N-1$ other electrons and the nuclei
(encapsulated in $V$). This is due to the fact that, in general, a Lorentz
invariant many--body Dirac equation does not exist if the Coulomb interaction is
taken to be the interaction potential between charges particles. In this work I
will not be explicitly taking into account relativistic retardation effects,
hence this treatment is not completely relativistic. This treatment does,
however, contain the chemically relevant relativistic effects such as
SOC and  scalar relativistic effects.
From the matrix representation of $\op{H}_D$, it is clear to see that it
acts upon a four--dimensional Hilbert space known as a bispinor field ,
$\ket{\Phi^{4c}}$, whose components are themselves two--dimensional,
\begin{equation}
\ket{\Phi^{4c}} = \begin{pmatrix}
 \ket{\Phi_L} \\ \ket{\Phi_S}
\end{pmatrix} \quad.
\end{equation}
Here, $\ket{\Phi_L}$ and $\ket{\Phi_S}$ are the so--called large and small
components of the bispinor respectively, and are both two--dimensional in the
spin manifold (i.e. spinors).

Although it is possible, in principle, to obtain the eigenstates of
\cref{eq:DiracHam} directly, it as often advantageous from a practical as well
as aesthetic perspective to transform the full four component relativistic
equations into a decoupled two component form. This is often convenient as the
resulting expressions closely resemble those found in non--relativistic
electronic structure theory and therefore allow the employment of standard
electronic structure methods with only minor modifications. In general, such a
transformation takes the form of a unitary operator, $\op{U}$, such that
\begin{equation}
\op{U}: 
\op{H}_D \mapsto \begin{pmatrix}
\op{H}_+ && \vc{0}_2 \\ \vc{0}_2 && \op{H}_- 
\end{pmatrix} \quad \text{ and } \quad
\ket{\Phi^{4c}} \mapsto \begin{pmatrix}
 \ket{\Phi^{2c}} \\ \vc{0}_2
\end{pmatrix} \quad.
\end{equation}
Thus making the transformed $\ket{\Phi^{2c}}$ an eigenstate of $\op{H}_+$.
The transformation under $\op{U}$, in effect, folds the contributions from
the small component of the wave function into the large components. In principle,
such a transformation is exact if a proper $U$ may be found. It is the case
that such an exact transformation is not practical to obtain due to the
effective many--body $V$, thus approximate decoupling schemes must be developed.
Several decoupling schemes have been explored in recent years, but in this work
I will only employ the use of the ``exact" two--component (X2C) method, for which
the exact details may be found elsewhere.

Using the X2C transformation method, we are able to obtain a total effecting
two--component Hamiltonian, which may be written in second quantized form,
\begin{equation}
\op{H}_\mathrm{X2C} = h_{pq}^\mathrm{X2C}c_p^\dagger c_q + 
  \frac{1}{2} \inner{pq}{rs} c_p^\dagger c_r^\dagger c_s c_q + V_\mathrm{NN}
  \label{eq:X2CHam}
\end{equation}
where $c_p$ and $c_p^\dagger$ are single spinor particle annihilation and
creation operators respectively. $\vc{h}^\mathrm{X2C}$ is the effective
relativistic core--Hamiltonian which includes both spin--orbit and scalar
relativistic effects, $\inner{\cdot}{\cdot}$, is the Coulomb operator expressed
in the basis of spinor MOs and $V_\mathrm{NN}$ is the nuclear--nuclear repulsion
energy. Expressing \cref{eq:X2CHam} in basis of a single Slater determinant and
expanding each spinor MO in the AO basis ($\lbrace \chi_\mu \rbrace$),
\begin{equation}
\phi_p ( \vc{r} ) = \sum_\mu \chi_\mu(\vc{r}) C_{\mu p}
\qquad
C_{\mu p} = \begin{pmatrix}
C_{\mu p}^\alpha \\ C_{\mu p}^\beta
\end{pmatrix}
\quad , \label{eq:AO2MO}
\end{equation}
we arrive at a relativistic two--component analogue of the Roothaan--Hall
equations familiar to traditional mean--field electronic structure theory,
\begin{align}
&\frac{1}{2}\left(
  F^S_{\mu\nu} \cdot \vc{I}_2 + F^k_{\mu\nu} \cdot \vc{\sigma_k}
\right) C_{\nu p} = (S_{\mu\nu}\cdot \vc{I}_2) C_{\nu p}
\epsilon_p\label{eq:Roothaan}
%\\
%\nonumber \\
%&F^S_{\mu\nu} = \left(
%  2\left( \mu\nu \vert \kappa\lambda \right) - 
%  \left(  \mu\lambda \vert \kappa\nu \right) 
%\right) P^S_{\lambda \kappa} \nonumber \\
%&F^k_{\mu\nu} = -\left(  \mu\lambda \vert \kappa\nu \right) P^k_{\lambda \kappa}\\
%\nonumber \\
%&\vc{P}^S = \frac{1}{2}\mathrm{Tr}_\sigma (\vc{P}\vc{I}_2) \nonumber \\
%&\vc{P}^k = \frac{1}{2}\mathrm{Tr}_\sigma (\vc{P}\vc{\sigma}_k)
%\label{eq:traceRel}\\
%&P_{\mu\nu} = C_{\mu i} C_{\nu i}^\dagger \nonumber
\end{align}
Here, $\vc{S}$ is the AO overlap matrix, 
%and $(\cdot \vert \cdot)$ is the
%Coulomb operator in the AO basis.
and $\{\epsilon_p\}$ is the set of MO
eigenenergies. $\vc{F}^S$,$\vc{F}^k$ and $\vc{P}^S,\vc{P}^k$ are the scalar and
vector parts of the spinor Fock and density matrices in the AO basis
respectively, 
%and are related to the full spinor operators in terms of the spin
%trace, $\mathrm{Tr}_\sigma$, relations of \cref{eq:traceRel}. 
Is important to note that $C_{\mu p} \in \mathbb{C}^2$ (where the $\alpha$ and
$\beta$ indicies represent inseparable spin-up and spin-down components of
$C_{\mu p}$), thus elements of the total spinor density (and thus the spinor
Fock) are 2x2 complex matrices.  \Cref{eq:Roothaan} may be solved
self--consistently to obtain the lowest energy (single) Slater determinantal
description of the many--body system.

The fact that one may write the X2C Hamiltonian in second quantized form makes it
suitable for application to post--self consistent field descriptions of
electronic correlation. Recently, I have extended the particle--particle
Tamm--Dancoff approximation to X2C optimized wave functions to study the fine
structure splittings in atomic and molecular systems, of which results are
presented in \cref{sec:pp-X2C}.


\linespread{1.0}
\bibliography{Journal_Short_Name,ppSH,Li_Group_References,GE,DBWY,Egidi_References}
\end{document}
