\documentclass[12pt]{article}
\usepackage{achemso}
\usepackage[margin=1.0in]{geometry}
\usepackage{setspace}
\usepackage{amsmath}             % for equation typesetting
\usepackage{amssymb}             % for equation typesetting
\usepackage{mathrsfs} 
\usepackage{wasysym}             % for geometric shapes
\usepackage{color}               % for colored fonts
\usepackage{setspace}            % for 1.5 and double spacing
\usepackage{graphicx}            % main graphics package
\usepackage{wrapfig}             % allow text wrapping around figures
\usepackage{subcaption}
%\doublespacing
\linespread{1.5}

\usepackage{appendix}

\usepackage[capitalize]{cleveref}
\crefname{figure}{Fig.}{Figs.}
\Crefname{figure}{Figure}{Figures}
\crefname{table}{Tab.}{Tabs.}
\Crefname{table}{Table}{Tables}
\crefname{equation}{Eq.}{Eqs.}
\Crefname{equation}{Equation}{Equations}
\crefname{section}{Sec.}{Secs.}
\Crefname{section}{Section}{Sections}
\crefname{appsec}{appendix}{appendices}
\Crefname{appsec}{Appendix}{Appendixes}

%\usepackage{titlesec}
%\titleformat*{\section}{\Large\bfseries}
%\titleformat*{\subsection}{\large\bfseries}

% Mathematical Shortcuts
\newcommand{\pfrac}[2]{\frac{\partial #1}{\partial #2}}   % partial derivative
\newcommand{\difrac}[2]{\frac{d #1}{d #2}}                % derivative
\newcommand{\bpar}[1]{\left( #1 \right)}                  % big parentheses
\newcommand{\bbra}[1]{\left[ #1 \right]}                  % big brackets
\newcommand{\bbar}[1]{\left| #1 \right|}                  % big bars
\newcommand{\bra}[1]{\left\langle #1 \right\vert}         % bra
\newcommand{\ket}[1]{\left\vert #1 \right\rangle}         % ket
\newcommand{\inner}[2]{\left\langle #1 \left\vert\right. #2 \right\rangle}            % bracket
\newcommand{\innerop}[3]{\left\langle #1 \left\vert #2 \right\vert #3 \right\rangle}  % operator matrix element
\newcommand{\innersub}[4]{\langle \bd{#1}_{#2}, \bd{#3}_{#4} \rangle}                 % bracket with subscripts
\newcommand{\half}{\frac{1}{2}}                           % 1/2
\newcommand{\powfrac}[3]{\bpar{\frac{#1}{#2}}^{#3}}       % fraction raised to a power
\newcommand{\ii}{\infty}                                  % infinity symbol
\newcommand{\tquad}{\quad\quad\quad}                      % triple-quad spacing
\renewcommand{\Im}{\text{Im}}                             % imaginary symbol
\renewcommand{\Re}{\text{Re}}                             % real symbol

\newcommand*\mycommand[1]{\texttt{\emph{#1}}}
\newcommand*\suchthat[0]{\text{ }\vert\text{ }}
\newcommand*\complexmatfield[1]{\mathbb{C}^{#1\text{x}#1}}
\newcommand*\speciallinear[2]{\mathrm{UnL}(#1,#2)}
\newcommand*\complexspeciallinear[1]{\speciallinear{#1}{\mathbb{C}}}
\newcommand*\generallinear[2]{\mathrm{GL}(#1,#2)}
\newcommand*\complexgenerallinear[1]{\generallinear{#1}{\mathbb{C}}}
\newcommand*\mat[1]{\boldsymbol{#1}}

\newcommand*\vc[1]{\boldsymbol{#1}}
\newcommand*\op[1]{\mathcal{#1}}


\title{General Exam}
\date{October 19, 2016 \\ CHB 239}
\author{David Williams-Young\\ Department of Chemistry, University of Washington}


\begin{document}
\linespread{1.0}
\maketitle
\linespread{1.5}

\newpage
\section{Introduction}

Formally spin--forbidden processes, such as intersystem crossing (ISC),  play
an important role in photochemistry. ISC is a transition beteen quantum states
of differing spin multiplicity which manifests either radiatively or
non-radiatively.  The functionality of many modern technologies depend heavily
on finely tuned ISC to control the rate of population and depopulation of
triplet intermediate states from singlet states upon photoexcitation. The fact
that the non--relativisitc treatment of molecular quantum mechanics predicts
phenomena such as ISC as formally forbidden, it has been long thought that the
time--scales at which they occur are far too long to compete with transitions
between states of the same spin multiplicity. However, recent studies have
shown that this is not necessarily the case, and that ISC can indeed be
competitive even at short time--scales.  ISC is an inheriently relativistic, and
time--dependant excited--state process. Thus to accurately treat these phenomena
theoretically, one must provide an \emph{ab initio} description of the
relativistic effects throughout the time--evolution of the quantum system.  

Relativistic effects, while often neglected in most standard treatments of
quantum mechanics, can have profound consequences in chemical
systems.\cite{Pyykko12_45} Scalar relativistic effects cause the contraction of
the core electron shells of heavy atoms, but perhaps of even more consequence is
the introduction of spin couplings in the Hamiltonian.  Spin-spin (SSC) and
spin-orbit (SOC) coupling can affect the electronic spin dynamics even in light
atoms. A direct consequence of these couplings on the electronic manifold is the
breaking of spin--symmetry as the Hamiltonian no longer commutes with the spin
operator, $\op{S}$. This breaking of spin--symmetry is what allows, at an
operator level, for formally spin--forbidden processes to occur, namely
ISC. Although some approaches have been purposed to account for SOC in the
treatment of the electronic manifold perturbatively\cite{Thiel14_JCP124101},
these approximations will break down whenever SOC is non--negligible. Hence a
variational, and well balanced approach much be adopted to account for these
interactions for the general case.  

In the quantum mechanical description of molecular systems, neglecting explicit
coupling to a quantized photon field, time evolution of the total molecular wave
function, $\ket{\Psi (t)}$ is governed by the Hamiltonian wave equation 
(in atomic units),
\begin{equation}
\op{H}(t) \ket{\Psi (t)} = i\partial_t \ket{\Psi(t)} \quad,
\label{eq:WaveEq}
\end{equation}
where $\op{H}(t)$ is the time--dependent Hamiltonian, $\partial_t$ is a partial
derivative with respect to time, and $t$ is a time parameter.  The moieties
enclosed in parentheses are taken to be parameters, i.e. the wave function is
parameterized by time.  In principle, one may solve (in some approximate manner)
\cref{eq:WaveEq} simultaneously for both the electronic and nuclear degrees of
freedom explicitly. This approach is, however, intractable for quantum systems
exceeding more than a few particles. To simplify the solutions of
\cref{eq:WaveEq}, one may formally decompose $\ket{\Psi (t)}$ into a product of
nuclear and electronic wave functions,
\begin{equation} 
\ket{\Psi (t)} = \ket{\Phi(\vc{R}(t),t)}\otimes\ket{\Theta(t)} 
\quad .  
\label{eq:exactSepElecNuc}
\end{equation} 
Here, $\ket{\Phi(\vc{R}(t),t)}$ and $\ket{\Theta (t)}$ are the electronic and
nuclear wave functions respectively and $\vc{R}(t)$ is the expectation value of
the nuclear position operator, $\vc{R}(t) =
\innerop{\Theta(t)}{\hat{\vc{R}}}{\Theta(t)}$.  
This formally exact treatment\cite{Gross10_PRL123002, Cederbaum08_JCP124101,
Ghosh15_MP1} allows one to, at least in in principle, bifurcate the treatment of
quantum molecular dynamics into explicit treatment of the electronic and nuclear
time dependence separately.  Thus, in order to theoretically treat molecular
dynamics accurately, one must be able to treat both the electronic and nuclear
degrees of freedom to some level of accuracy.

When obtaining the time evolution of the total molecular wave function, it is
often advantageous to work in so called adiabatic basis (the eigenstates of
\cref{eq:WaveEq}) of quantum states,
\begin{equation}
\op{H}(t) \ket{\Psi_I (t)} = E_I(t) \ket{\Psi_I (t)}
\quad \forall t, I \in \mathbb{N},
\label{eq:EigSpec}
\end{equation}
where $\ket{\Psi_I}$ and $E_I$ represent the $I$-th adiabatic wave function and
eigenenergie, respectively. For the purposes of the current discussion, we will
be assuming a discrete eigenspectrum of the Hamiltonian that is smooth over $t$.
Taking the separation of the total wave function in \cref{eq:exactSepElecNuc},
we may rewrite the total wave function as a linear combination of adiabatic
states,
\begin{equation}
\ket{\Psi (t)} = c_I \ket{\Phi_I (\vc{R}(t),t)} \otimes \ket{\Theta_I (t)}
\quad ,
\label{eq:AdiaExp}
\end{equation}
where $\{ c_I \}$ is a set of expansion coefficients, and $\{\ket{\Phi_I}\}$ and
$\{\ket{\Theta_I}\}$ are the sets of electronic and nuclear adiabatic wave
functions, respectively.. Given the entire eigenspectrum of \cref{eq:EigSpec},
the expansion in \cref{eq:AdiaExp} is formally exact, thus the problem now
becomes how to properly obtain the adiabatic electronic and nuclear wave
functions.

There exsit a vast plethora of computational methods to treat molecular dynamics
based on the separation in \cref{eq:exactSepElecNuc}. For the purposes of this
work, the discussion will be restricted to taht of trajectory surface hopping
(TSH).  TSH is one of the most widely applied methods for simulating
electronically non-adiabatic dynamics of molecular and condensed-phase
systems.\cite{Barbatti11_1759, Tavernelli14_62, Tully12_22A301, Tully98_407,
Hynes14_97} At its core, TSH is a stochastic algorithm that controls which
electronic state dictates the forces on the nuclei during a semi-classical
molecular dynamics simulation.\cite{Preston71_562} As described in the seminal
works on the method\cite{Tully98_407, Tully90_1061}, via insertion of
\cref{eq:AdiaExp} into \cref{eq:WaveEq}, one may derive an approximate equation
of motion for an electronic superposition state evolving along a classical
nuclear trajectory,
\begin{align}
  &i  \partial_t c_K(t) = H_{KJ}(t) c_J(t) \label{eq:EOMCs} \\
  &H_{KJ}(t) = \delta_{KJ}E_J(\vc{R}(t)) - i d_{KJ}^\xi (t) \partial_t R_{\xi}(t) \label{eq:HamiltonianElec} \\
  &d_{KJ}^\xi (t) = \innerop{\Phi_K(\vc{R}(t),t)}{\nabla^\xi}{\Phi_J(\vc{R}(t),t)} \label{eq:NAC}
  \quad .
\end{align}
The nuclear position time evolution is governed by Newtonian mechanics,
\begin{equation}
  -\nabla E_c(t) = \vc{m}\ddot{\vc{R}}(t) \label{eq:Newton}
  \quad.
\end{equation}
While $E_J$ in \cref{eq:HamiltonianElec} represents an arbitrary energy
eigenvalue of adiabat $J$, $E_c(t)$ ($c$ indicating ``current") from
\cref{eq:Newton} is the electronic eigenenergy designated by the TSH algorithm
to drive the nuclear evolution at time $t$.  $\vc{m}$ collects the masses for
each nucleus, and $d_{IJ}^\xi$ is the rank-3 non--adiabatic coupling (NAC)
tensor that provides an affine topological connection on the electronic manifold
through the nuclear momentum operator. $\xi$ is an arbitrary nuclear coordinate.

%Talk about algorithm to switch the state throughout the dynamics

The representation of the separated equations of motion
in \cref{eq:EOMCs,eq:HamiltonianElec,eq:NAC,eq:Newton} are completely general to
any method used to obtain the electronic adiabatic states. There have been many
extensions of this methods to a vast number of different electronic structure
methods. Recently, I have extended this method for use within the
particle--particle Tamm--Dancoff approximation for the description of the
electronic states\cite{DBWY16_Submitted1}, of which results are presented in
\cref{sec:pp-TSH}. Although TSH has been applied to study ISC using perturbative
approaches of accounting for SOC effects, I argue that this treatment is
inherently flawed as the SOC that is argued to be so negligible as to be treated
perturbatively is exactly the term that gives rise to the physical phenomena
that they observe. To properly treat SOC in TSH, one must account for the SOC in
a non--perturbative manner through an \emph{ab initio} treatment of some
relativistic Hamiltonian.

test\cite{DBWY16_JCTC935}
\linespread{1.0}
\bibliography{Journal_Short_Name,ppSH,Li_Group_References,GE,DBWY,Egidi_References}
\end{document}
