\section{Introduction}

Formally spin--forbidden processes, such as intersystem crossing (ISC),  play an
crucial role in photochemistry.\cite{Marian_SOC,Scaiano_Photo,Steer93_CR67} ISC
is a transition between quantum states of differing spin multiplicity which
manifests either radiatively or non-radiatively.  The functionality of many
modern technologies, such as phosphorescent organic light emitting diodes
(PHOLEDs)\cite{Miyaguchi99_JAP1502},depend heavily on finely tuned ISC to
control the rate of population and depopulation of triplet intermediate states
from singlet states upon photoexcitation. The fact that the non--relativistic
treatment of molecular quantum mechanics predicts phenomena such as ISC as
formally forbidden, it has been long thought that the time--scales at which they
occur are far too long to compete with transitions between states of the same
spin multiplicity. However, recent studies have shown that this is not
necessarily the case, and that ISC can indeed be competitive even at short
time--scales.\cite{Marian_SOC,Li16_JA2}  ISC is an inherently relativistic, and
non--adiabatic time--dependent process\cite{Dyall07_book,Reiher15_book}.  Thus
to accurately treat these phenomena theoretically, one must provide an \emph{ab
initio} description of the relativistic effects throughout the non--adiabatic
time--evolution of the quantum system.  {\bf While much effort has been directed
at the accurate treatment of non--adiabatic molecular dynamics, the scope of
inquiry has been primary limited to non--relativistic regime.}

Relativistic effects, while often neglected in most standard treatments of
quantum mechanics, can have profound consequences in chemical
systems.\cite{Pyykko12_45} Scalar relativistic effects cause the contraction of
the core electron shells of heavy atoms, but perhaps of even more consequence is
the introduction of spin couplings in the Hamiltonian.  Spin-spin (SSC) and
spin-orbit (SOC) coupling can affect the electronic spin dynamics even in light
atoms. 
%A direct consequence of these couplings on the electronic manifold is the
%breaking of spin--symmetry as the Hamiltonian no longer commutes with the spin
%operator, $\op{S}$. 
This introduction of spin--couplings is what allows, at an
operator level, for formally spin--forbidden processes to occur, namely
ISC. 
%Although some approaches have been purposed to account for SOC in the
%treatment of the electronic manifold perturbatively\cite{Thiel14_JCP124101},
%these approximations will break down whenever SOC is non--negligible. 
%Further,
%these treatments neglect any contributions from many--body SOC or SSC, which
%must be present for a proper treatment of spin interaction. 
%Hence a
%relativistic, variational, and well balanced approach much be adopted to account
%for these interactions for the general case. 
There exist several \emph{ab
initio} methods to approximately treat these relativistic effects, however, in
this work the discussion will be limited to two--component relativistic
Hamiltonians, specifically the ``exact two--component" (X2C) Hamiltonian.
\cite{Liu05_241102,Peng06_044102,Saue07_064102,Peng09_031104,Reiher13_184105,Cheng07_104106,Liu16_204}
This choice of two--component transformation is due to its simplicity and
accuracy.  A brief overview of the X2C method may be found in \cref{sec:X2C}.

In the quantum mechanical description of molecular systems, time evolution of
the total molecular wave function, $\ket{\Psi (t)}$ is governed by the
Hamiltonian wave equation (in atomic units),
\begin{equation}
\op{H}(t) \ket{\Psi (t)} = i\partial_t \ket{\Psi(t)} \quad,
\label{eq:WaveEq}
\end{equation}
where $\op{H}(t)$ is the time--dependent Hamiltonian, $\partial_t$ is a partial
derivative with respect to time, and $t$ is a time parameter.  In principle, one
may solve \cref{eq:WaveEq} (in some approximate manner) simultaneously for both
the electronic and nuclear degrees of freedom explicitly. However, this approach
is  intractable for quantum systems exceeding more than a few particles. To
simplify the time integration of \cref{eq:WaveEq}, one may formally decompose
$\ket{\Psi (t)}$ into a product of nuclear and electronic wave functions,
\begin{equation} 
\ket{\Psi (t)} = \ket{\Phi(\vc{R}(t),t)}\otimes\ket{\Theta(t)} 
\quad .  
\label{eq:exactSepElecNuc}
\end{equation} 
Here, $\ket{\Phi(\vc{R}(t),t)}$ and $\ket{\Theta (t)}$ are the electronic and
nuclear wave functions respectively and $\vc{R}(t)$ is the expectation value of
the nuclear position operator, $\vc{R}(t) =
\innerop{\Theta(t)}{\hat{\vc{R}}}{\Theta(t)}$.  
This formally exact treatment\cite{Gross10_PRL123002, Cederbaum08_JCP124101,
Ghosh15_MP1} allows one to, at least in in principle, bifurcate the treatment of
quantum molecular dynamics into explicit treatment of the electronic and nuclear
time dependence separately.  

When obtaining the time evolution of the total molecular wave function, it is
often advantageous to work in so called adiabatic basis of quantum states,
\begin{equation}
\op{H}(t) \ket{\Psi_I (t)} = E_I(t) \ket{\Psi_I (t)}
\quad \forall t, I \in \mathbb{N},
\label{eq:EigSpec}
\end{equation}
where $\ket{\Psi_I}$ and $E_I$ represent the $I$-th adiabatic wave function and
eigenenergie, respectively.  Taking the separation of the total wave function in
\cref{eq:exactSepElecNuc}, we may rewrite the total wave function as a linear
combination of adiabatic states,
\begin{equation}
\ket{\Psi (t)} = c_I \ket{\Phi_I (\vc{R}(t),t)} \otimes \ket{\Theta_I (t)}
\quad ,
\label{eq:AdiaExp}
\end{equation}
where $\{ c_I \}$ is a set of expansion coefficients, and $\{\ket{\Phi_I}\}$ and
$\{\ket{\Theta_I}\}$ are the sets of electronic and nuclear adiabatic wave
functions, respectively. Given a complete eigenspectrum of \cref{eq:EigSpec},
the expansion in \cref{eq:AdiaExp} is formally exact, thus the problem now
becomes how to properly obtain the adiabatic electronic and nuclear wave
functions.

Over the years, perhaps the most successful \emph{ab initio} treatment of the
adiabatic electronic wave function has been in the configuration interaction
(CI) expansion of the electronic wave function.
\cite{Meyer00_PR1,Knowles88_JCP1,Handy82_JCP1,Saalfrank04_JCP124102,
Schlegel92_CPL524,Handy84_CPL315,Joergensen78_JCP3833,Fleig08_JCP014108}
A brief overview of the CI method may be found in \cref{sec:ci}.  Many
successful attempts have been made to extend CI variants to relativistic
Hamiltonians\cite{Neese13_JCP104113, Olsen97_TCA125, Jensen96_JCP4083,
Fleig08_JCP014108}, including my recent work on the extension of the
particle--particle Tamm-Dancoff approximation (pp-TDA) to the X2C method
\cite{DBWY16_Accepted1} of which results are presented in \cref{sec:pp-X2C}. The
CI method is simplest method, in formalism, for the addition of explicit
many--body interaction back into mean--field description of the ground state
electronic wave function such as Hartree--Fock. In addition to correcting the
uncorrelated behavior of mean--field ground states, the CI method is also able
to treat excited electronic states in a correlated manner, making it a suitable
method for use in non--adiabatic molecular dynamics.

There exist a vast plethora of computational methods to treat non--adiabatic
molecular dynamics based on the separation in
\cref{eq:exactSepElecNuc,eq:AdiaExp}.
\cite{Meyer00_PR1,Li11_144102,Li05_084106,Preston71_562}
For the purposes of this work, the discussion will be restricted to that of the
trajectory surface hopping (TSH) method.  TSH is one of the most widely applied
methods for simulating electronically non-adiabatic dynamics of molecular and
condensed-phase systems.\cite{Barbatti11_1759, Tavernelli14_62, Tully12_22A301,
Tully98_407, Hynes14_97} At its core, TSH is a stochastic algorithm that
controls which electronic adiabat dictates the forces on the nuclei during a
classical nuclear trajectory.\cite{Preston71_562} As a semi-classical method,
the time--evolution of the electronic degrees of freedom are treated as quantum
mechanical while the time evolution of the nuclear positions are treated as
classical through Newton's equations of motion. 
%In essence, this classical treatment may be
%thought of as the nuclear wave function begin written as a product of delta
%functions centered at the classical nuclear positions
%\begin{equation}
%\inner{\vc{R}}{\Theta_I (t)} = 
%  \prod_{A = 0}^{N_\mathrm{atoms}} \delta^3(\vc{R} - \vc{R}_A(t))
%  \quad \forall I,
%  \label{eq:ClassicalNuclei}
%\end{equation}
%where $N_\mathrm{atoms}$ is the number of nuclei in the molecule.  
TSH provides an ideal algorithm for the simulation and interpretation of many
non--adiabatic processes, such as ISC, as it is blind to the underlying method
used to obtain the electronic adiabats. Thus it is general to cases where the
coupling between electronic adiabats is large or small, all while allowing
smooth transitions between them. A brief overview of the TSH algorithm may be
found in \cref{sec:TSH}. 


The quality of a TSH simulation is heavily dependent on the quality of the
underlying description of the electronic potential energy surface.
Traditionally, CI variants have been the most often applied basis for TSH
implementations and their use is well documented in the literature.
\cite{ Gonzalez11_JCTC1253, Thiel14_JCP124101, Lischka10_8778, Martinez14_1,
Robb02_10494, Gonzalez14_JCP204302}
Recently, I
have extended TSH for use within the pp-TDA \cite{DBWY16_Submitted1}, of which
results are presented in \cref{sec:pp-TSH}.  While some attempts have been made
to study ISC using TSH via a perturbative inclusion of SOC in the CAS
description of electronic adiabats
\cite{Thiel14_JCP124101,Gonzalez14_JCP204302,Gonzalez11_JCTC1253}, no attempts
have been made to provide an explicitly relativistic treatment of these effects
within TSH. These perturbative approaches are valid only in the limit of
negligible SOC, which seems to contradict the premise as it is in the SOC that
ISC is made possible. This lack of relativistic treatment of SOC in TSH is an
inherent flaw in current methods to model ISC and indicates a need for the
development of new theoretical methods to model the electronic adiabatic states
relativistically.
{\bf 
In this work, I will outline my proposed research to accurately simulate
\emph{ab initio} relativistic molecular dynamics using trajectory surface
hopping with relativistic CI wave functions.
}

